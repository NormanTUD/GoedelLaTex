\newlength{\spacebetweenbreakedequations}
\setlength{\spacebetweenbreakedequations}{-1.1cm}
\newlength{\spaceafterbreakedequation}
\setlength{\spaceafterbreakedequation}{-0.7cm}

\emergencystretch10em

\usepackage[utf8]{inputenc}
\usepackage[T1]{fontenc}
\usepackage{fourier}
\usepackage[ngerman]{babel}
\usepackage{soulutf8}
\usepackage{amsmath}
\usepackage{pifont}
\usepackage{amssymb}
\usepackage{graphicx}
\usepackage{enumerate}
\usepackage{amssymb}
\usepackage{mathtools}
\usepackage{xargs}
\usepackage[pdftex,dvipsnames]{xcolor}
\usepackage{amsfonts,amsthm}
\usepackage{ifthen}
\usepackage{xpatch}
\usepackage{url}
\usepackage{lettrine}

\setcounter{DefaultLines}{4}
\setlength{\DefaultFindent}{0.5em}
\setlength{\DefaultNindent}{0em}
\renewcommand{\LettrineFontHook}{\usefont{U}{yinit}{m}{n}}
\renewcommand{\DefaultLoversize}{-0.65}


\def\middlebreak {\nulldelimiterspace0pt
\right.\allowbreak\mskip 0mu plus .5mu \nulldelimiterspace0pt\left.}%

\makeatletter
\newcommand{\customlabel}[2]{%
\protected@write \@auxout {}{\string \newlabel {#1}{{#2}{}}}}
\makeatother

\usepackage[colorinlistoftodos,prependcaption,textsize=tiny]{todonotes}

\everymath{\displaystyle}
\allowdisplaybreaks

\ifodd\value{showComments}
	\usepackage[a3paper,
		inner=20mm,
		outer=171mm,% = marginparsep + marginparwidth
		top=20mm,
		bottom=25mm,
		marginparsep=5mm,
		marginparwidth=160mm,
		paperheight=27cm
		%showframe% for show your page design, normaly not used
	]{geometry}
	\makeatletter
	\newcommand{\mnote}[1]{%
		\marginpar{% 
			\textcolor{darkgray}{%
				\small\arabic{commentaryNumber}: #1}%
		}%
		\stepcounter{commentaryNumber}%
	}
	\makeatother
\else 
	\newcommand{\mnote}[1]{}
\fi

\newcommand{\refsatz}[1]{\textcolor{orange}{\ref{#1} (#1)}}

\DeclarePairedDelimiter{\ceil}{\lceil}{\rceil}

\newcommand{\fnelfa}{11a}
\newcommand{\fnachtzehna}{18a}
\newcommand{\fnvierunddreissiga}{34a}
\newcommand{\fnvierunddreissigb}{34b}
\newcommand{\fnfunfundvierziga}{45a}
\newcommand{\fnachtundvierziga}{48a}

%\newcommand{\mnote}[1]{\marginpar{\todo{\tiny#1}}}
\newcommand{\mcode}[1]{\textcolor{gray}{\texttt{#1}}}

\makeatletter
\newcommand{\Spvek}[2][r]{%
  \gdef\@VORNE{1}
  \left(\hskip-\arraycolsep%
    \begin{array}{#1}\vekSp@lten{#2}\end{array}%
  \hskip-\arraycolsep\right)}

\def\vekSp@lten#1{\xvekSp@lten#1;vekL@stLine;}
\def\vekL@stLine{vekL@stLine}
\def\xvekSp@lten#1;{\def\temp{#1}%
  \ifx\temp\vekL@stLine
  \else
    \ifnum\@VORNE=1\gdef\@VORNE{0}
    \else\@arraycr\fi%
    #1%
    \expandafter\xvekSp@lten
  \fi}
\makeatother

\DeclareRobustCommand*{\pmstar}{%
  \text{%
      \resizebox{!}{.75\height}{\ding{107}}%
        }%
}

\ifodd\value{leftRightBracketsColor}
	\makeatletter
	\newcount\bracketnum
	\newcommand\makecolorlist[1]{%
	    \bracketnum0\relax
		\makecolorlist@#1,.%
		    \bracketnum0\relax
	}
	\def\makecolorlist@#1,{%
	    \advance\bracketnum1\relax
		\expandafter\def\csname bracketcolor\the\bracketnum\endcsname{\color{#1}}%
		    \@ifnextchar.{\@gobble}{\makecolorlist@}%
	}
	\let\oldleft\left
	\let\oldright\right
	\def\left#1{%
	    \global\advance\bracketnum1\relax 
		\colorlet{temp}{.}%
		    \csname bracketcolor\the\bracketnum\endcsname
			\oldleft#1%
			    \color{temp}%
	}
	\def\right#1{%
	    \colorlet{temp}{.}%
		\csname bracketcolor\the\bracketnum\endcsname
		    \oldright#1%
			\global\advance\bracketnum-1\relax
			    \color{temp}%
	}
	\makeatother
	\makecolorlist{green, cyan, blue, violet, magenta, pink, teal, lime, orange, yellow, purple, red}
\else
	\ifodd\value{leftRightBrackets}
	\else
		\def\left#1{#1}
		\def\right#1{#1}
	\fi
\fi
