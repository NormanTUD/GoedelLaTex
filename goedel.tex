\documentclass[draft]{scrartcl}

\newcounter{showComments}
\setcounter{showComments}{1}

\newcounter{commentaryNumber}
\setcounter{commentaryNumber}{1}

\newcounter{leftRightBrackets}
\setcounter{leftRightBrackets}{1}

\newcounter{leftRightBracketsColor}
\setcounter{leftRightBracketsColor}{1}

\newlength{\spacebetweenbreakedequations}
\setlength{\spacebetweenbreakedequations}{-1.1cm}
\newlength{\spaceafterbreakedequation}
\setlength{\spaceafterbreakedequation}{-0.7cm}

\emergencystretch10em

\usepackage[utf8]{inputenc}
\usepackage[T1]{fontenc}
\usepackage{fourier}
\usepackage[ngerman]{babel}
\usepackage{soulutf8}
\usepackage{amsmath}
\usepackage{pifont}
\usepackage{amssymb}
\usepackage{graphicx}
\usepackage{enumerate}
\usepackage{amssymb}
\usepackage{mathtools}
\usepackage{xargs}
\usepackage[pdftex,dvipsnames]{xcolor}
\usepackage{amsfonts,amsthm}
\usepackage{ifthen}
\usepackage{xpatch}
\usepackage{url}
\usepackage{lettrine}
\usepackage{marginnote}

\usepackage[
	citestyle=authortitle-ibid,
	isbn=true,
	url=true,
	backref=true,
	backrefstyle=none,
	pagetracker=true,
	maxbibnames=50,
	defernumbers=true,
	maxcitenames=10,
	backend=bibtex,
	urldate=comp,
	dateabbrev=false,
	sorting=nty,
	ibidtracker=true,
	backend=biber
]{biblatex}
\bibliography{literatur.bib}

\renewcommand\neq{\mathrel{\vphantom{|}\mathpalette\xsneq\relax}}
\newcommand\xsneq[2]{%
  \ooalign{\hidewidth$#1|$\hidewidth\cr$#1=$\cr}%
}

\setcounter{DefaultLines}{4}
\setlength{\DefaultFindent}{0.5em}
\setlength{\DefaultNindent}{0em}
\renewcommand{\LettrineFontHook}{\usefont{U}{yinit}{m}{n}}
\renewcommand{\DefaultLoversize}{-0.65}


\def\middlebreak {\nulldelimiterspace0pt
\right.\allowbreak\mskip 0mu plus .5mu \nulldelimiterspace0pt\left.}%

\makeatletter
\newcommand{\customlabel}[2]{%
\protected@write \@auxout {}{\string \newlabel {#1}{{#2}{}}}}
\makeatother

\usepackage[colorinlistoftodos,prependcaption,textsize=tiny]{todonotes}

\everymath{\displaystyle}
\allowdisplaybreaks

\ifodd\value{showComments}
	\usepackage[a3paper,
		inner=20mm,
		outer=27cm,% = marginparsep + marginparwidth
		top=20mm,
		bottom=25mm,
		marginparsep=5mm,
		marginparwidth=24cm,
		paperheight=29.7cm,
		paperwidth=42cm
		%showframe% for show your page design, normaly not used
	]{geometry}
	\makeatletter
	\newcommand{\mnote}[1]{%
		\marginnote{%
			\textcolor{darkgray}{%
				\small\arabic{commentaryNumber}: #1}%
		}%
		\stepcounter{commentaryNumber}%
	}
	\makeatother
\else
	\newcommand{\mnote}[1]{}
\fi

\newcommand{\refsatz}[1]{\textcolor{orange}{\ref{#1} (#1)}}

\DeclarePairedDelimiter{\ceil}{\lceil}{\rceil}

\newcommand{\fnelfa}{11a}
\newcommand{\fnachtzehna}{18a}
\newcommand{\fnvierunddreissiga}{34a}
\newcommand{\fnvierunddreissigb}{34b}
\newcommand{\fnfunfundvierziga}{45a}
\newcommand{\fnachtundvierziga}{48a}

%\newcommand{\mnote}[1]{\marginpar{\todo{\tiny#1}}}
\newcommand{\mcode}[1]{\textcolor{gray}{\texttt{#1}}}

\makeatletter
\newcommand{\Spvek}[2][r]{%
  \gdef\@VORNE{1}
  \left(\hskip-\arraycolsep%
    \begin{array}{#1}\vekSp@lten{#2}\end{array}%
  \hskip-\arraycolsep\right)}

\def\vekSp@lten#1{\xvekSp@lten#1;vekL@stLine;}
\def\vekL@stLine{vekL@stLine}
\def\xvekSp@lten#1;{\def\temp{#1}%
  \ifx\temp\vekL@stLine
  \else
    \ifnum\@VORNE=1\gdef\@VORNE{0}
    \else\@arraycr\fi%
    #1%
    \expandafter\xvekSp@lten
  \fi}
\makeatother

\DeclareRobustCommand*{\pmstar}{%
  \text{%
      \resizebox{!}{.75\height}{\ding{107}}%
        }%
}

\ifodd\value{leftRightBracketsColor}
	\makeatletter
	\newcount\bracketnum
	\newcommand\makecolorlist[1]{%
	    \bracketnum0\relax
		\makecolorlist@#1,.%
		    \bracketnum0\relax
	}
	\def\makecolorlist@#1,{%
	    \advance\bracketnum1\relax
		\expandafter\def\csname bracketcolor\the\bracketnum\endcsname{\color{#1}}%
		    \@ifnextchar.{\@gobble}{\makecolorlist@}%
	}
	\let\oldleft\left
	\let\oldright\right
	\def\left#1{%
	    \global\advance\bracketnum1\relax
		\colorlet{temp}{.}%
		    \csname bracketcolor\the\bracketnum\endcsname
			\oldleft#1%
			    \color{temp}%
	}
	\def\right#1{%
	    \colorlet{temp}{.}%
		\csname bracketcolor\the\bracketnum\endcsname
		    \oldright#1%
			\global\advance\bracketnum-1\relax
			    \color{temp}%
	}
	\makeatother
	\makecolorlist{green, cyan, blue, violet, magenta, pink, teal, lime, orange, yellow, purple, red}
\else
	\ifodd\value{leftRightBrackets}
	\else
		\def\left#1{#1}
		\def\right#1{#1}
	\fi
\fi



\begin{document}

\pagenumbering{Roman}

\let\originalthefootnote\thefootnote
\renewcommand*{\thefootnote}{\fnsymbol{footnote}}

\centeredquote{Das Muster einer vollständigen \so{Fiction} ist die Logik.
Hier wird ein Denken \so{erdichtet}, wo ein Gedanke als Ursache eines anderen 
Gedankens gesetzt wird; alle Affekte, alles Fühlen und Wollen wird hinweg 
gedacht. Es kommt dergleichen in der Wirklichkeit nicht vor: diese ist 
unsäglich anders complicirt. Dadurch daß wir jene Fiction als Schema 
anlegen, also das thatsächliche Geschehen beim Denken gleichsam durch
einen Simplificationsapparat \so{filtriren}: bringen wir es zu einer 
\so{Zeichenschrift} und \so{Mittheilbarkeit} und \so{Merkbarkeit} der logischen Vorgänge.
Also: das geistige Geschehen zu betrachten, \so{wie als ob es dem Schema jener 
regulativen Fiction entspräche}: dies ist der Grundwille. Wo es „Gedächtniß“ 
giebt, hat dieser Grundwille gewaltet. — In der Wirklichkeit giebt es kein 
logisches Denken, und kein Satz der Arithmetik und Geometrie kann aus ihr 
genommen sein, weil er gar nicht vorkommt.\\
Ich stehe anders zur Unwissenheit und Ungewißheit. Nicht, daß etwas
unerkannt bleibt, ist mein Kummer; ßso{ich freue mich}, daß es vielmehr eine 
Art von Erkenntniß geben \so{kann} und bewundere die Complicirtheit dieser 
Ermöglichung. Das Mittel ist: die Einführung vollständiger Fictionen 
als Schemata, nach denen wir uns das geistige Geschehen einfacher denken 
als es ist. Erfahrung ist nur möglich mit Hülfe von Gedächtniß: 
Gedächtniß ist nur möglich vermöge einer Abkürzung eines geistigen Vorgangs zum \so{Zeichen}.\\\\
Die Zeichenschrift.\\\\
\so{Erklärung}: das ist der Ausdruck eines neuen Dinges vermittelst
der Zeichen von schon bekannten Dingen.}{Friedrich Nietzsche\footcite[§ 249]{fragmente85}}

\section*{Einleitung}

\subsection*{Algorithmen}

\lettrine[nindent=0em]{\color{purple}A}{lgorithmen} sind endliche Abfolgen von\mnote{Für
die Spezifikationen der Schritte von Algorithmen siehe Knuth, \citetitle{taocp1}, S. 4f. Die Punkte sind:
\begin{itemize}
	\item Finiteness
	\item Defineteness
	\item Input
	\item Output
	\item (Effectiveness; eingeklammert, weil Knuth Computer und praktische
		Algorithmen im Kopf hat; \textsc{Gödel}s Beweise aufzuschreiben wäre alles andere als effizient; es ist
		aber \textit{prinzipiell} möglich.)
\end{itemize}}
genau spezifizierten Schritten, um ein mathematisches oder logisches Problem zu lösen.
Die Frage, vor der \textsc{Hilbert} stand, lautete: wenn man die Mathematik formalisiert,
sind dann alle wahren Aussagen durch einen Algorithmus beweisbar? Oder: kann man durch
einen Algorithmus, der alle möglichen Aussagen durchschreitet, von jeder Einzelnen
in endlicher Zeit herausfinden, ob sie wahr ist oder nicht?

Für \textsc{Hilbert} schien die Sache klar zu sein: es musste so sein. Nur fehlte ihm der Beweis
dafür. Erst \textsc{Gödel} zeigte, dass die Sache viel vertrackter ist und so nicht geht.

\subsection*{Beweisbarkeit und \textsc{Gödel}nummern}

Was heißt \frq beweisbar\flq? Beweise sind Zurückführungen auf einzelne Axiome mithilfe
von syntaktischen Veränderungsregeln.\mnote{Zumindest lassen sie sich analog zu
solchen Regeln behandeln, wie die moderne Computertechnik eindrucksvoll zeigt. Vgl. Kommentar \ref{tractatus}\nocite{wittgenstein}. Vgl.
auch \citeurl{altenkirch} für ein Weiterdenken in eine alternative Richtung, die
aber auch das \frq \textsc{Gödel}problem\flq\ hat. Eine grundlegend andere Alternative, in 
der die Gödelisierung notwendigerweise scheitert, ist der Ultrafinitismus. Mit einer begrenzten
Menge an $\mathbb{N}$ ist es unmöglich, alle Sätze auch nur prinzipiell zu formalisieren. Der Ultrafinitismus
ist jedoch nur schwach begründet.}
So kann man jede mathematische Aussage als Zeichenkette
auffassen und schauen, ob man -- durch erlaubte Veränderungen der Axiome des Systems --
irgendwie auf diese Zeichenkette kommt. Hierbei ist \textsc{Gödel} ein großer Vorreiter der
modernen Informatik, denn Computer können auch nur nach fest definierten Regeln
Zeichenketten bearbeiten. Die mathematische Formel wird dabei nur sekundär
inhaltlich\mnote{Die inhaltliche Betrachtung findet erst
beim Menschen statt, der versucht, das, was die Formel aussagt, zu verstehen. Der Formel selbst
liegt diese inhaltliche Deutung nicht bei, was unter Anderem zur extrem breiten Verwendbarkeit
der Mathematik folgt. Vereinfacht gesagt: mit der gleichen Formel kann man berechnen, wie ein Atomkraftwerk funktioniert,
aber auch eine Atombombe; zwei sehr unterschiedliche Anwendungsbereiche mit ähnlicher mathematische Struktur.}
betrachtet und primär als Ansammlung von Zeichen, die nach Regeln bearbeitet werden.
Jede bewiesene mathematische Aussage lässt sich prinzipiell als sehr lange Zeichenkette
darstellen, die nur aus den Grundoperationen besteht (z.\,B. XOR, AND und NOT in
der modernen Computertechnik). Daraus lassen sich Operationen wie das Addieren
herleiten, woraus sich das Multiplizieren herleiten lässt (mit ganzen Zahlen) usw.
Aber das sind alles nur Vereinfachungen, damit wir als Menschen, die damit arbeiten,
besser damit klarkommen. Inhaltlich ist ein $a + b$ gleichbedeutend mit einer langen
Kette an UND, XOR und NICHT-Operationen im CPU eines Computers.

\textsc{Gödel} zeigte, dass es nicht möglich ist, alle wahren Aussagen zu beweisen,
in dem er einen Algorithmus anbietet, der -- komplett zurückführbar auf die logischen
Grundaussagen --, immer Sätze konstruieren kann, die zwar durch metamathematische
Überlegungen von \frq außerhalb\flq\ des Systems\mnote{Vgl. Kommentar \ref{metamatematischeueberlegungen}.}
als geltend gezeigt werden können, die aber nicht innerhalb des Systems beweisbar sind.

\subsection*{Die \text{\citefield{principiamathematicavol1f}{title}} und die Typentheorie}

\textsc{Russell} und \textsc{Whitehead} haben erkannt, dass man solche Sätze mit selbstreferenziellen
Aussagen generieren kann, z.\,B. \frq dieser Satz ist falsch\flq,\mnote{\customlabel{luegezwei}{\arabic{commentaryNumber}}
Kommentar \ref{luegeeins} ist falsch.} und daher haben sie
in der Principia Mathematica solche Sätze, die sich auf sich selbst beziehen,
\frq verboten\flq\mnote{$\longrightarrow$ Typentheorie, vgl. \citefield{septypetheory}{title} (\citefield{septypetheory}{howpublished}).}.
Aber selbst, wenn man das verbietet, lassen sich dennoch ver\-gleich\-bare Sätze
konstruieren.

\subsection*{(Vereinfachtes) Beispiel für einen gödelisierten Satz}

Wieder ein vereinfachtes Beispiel: wir nehmen an, dass jeder mit den Grundzeichen
der Mathematik definierbare Satz eine Nummer bekommt. Dort sind nun alle nur überhaupt
definierbaren Sätze, sinnvoll wie sinnlos, falsch wie richtig,
wie z.\,B. \frq $45 + 16 = 1$\flq, \frq $1 + 1 = 2$\flq, aber auch \frq der Satz mit
der Nummer 4922\footnote{Die Nummern sind nur beispielhaft gewählt und entsprechen
nicht der wirklichen \textsc{Gödel}isierung der Sätze.}
ist nicht beweisbar\flq. Wenn aber letzterer Satz gerade
\frq zufällig\flq\ tatsächlich die zugewiesene Nummer 4922 hat, dann behauptet
der Satz seine eigene Unbeweisbarkeit, und das mehr oder weniger
zufällig, ohne dass eine selbstreferenzielle-Strukturen-verbietende
Typentheorie das verbieten könnte.\mnote{Siehe dazu Fußnote \ref{zufaellig}.}

Unbeweisbar heißt hier, dass der Satz nicht hergeleitet werden kann aus den gewählten
Axiomen. Aber durch metamathematische Überlegungen können wir ihn doch als wahr anerkennen,
weil er eben nicht aus den Axiomen herleitbar ist (d.\,h. unbeweisbar, was der
Satz ja auch behauptet).

Das letzte Ressort könnte sein, ihn als Axiom zu setzen. Aber da alle potenziell
möglichen Sätze generiert werden, dürfen wir den Satz mit der Nummer 559321 nicht
aus den Augen verlieren, er besagt: \frq der Satz mit der Nummer 559321 ist
nicht beweisbar\flq\mnote{Dieser Satz könnte auch aussagen: $\thicksim \thicksim \thicksim \text{Bew}(4922) $, wenn
man mehr als die absolute Minimalmenge nimmt. Vgl. dazu Kommentar \ref{minimalmenge}. Aber das würde die
Sache nur komplizierter machen.}. Wir
stehen also wieder vor dem gleichen Problem und müssten
wieder ein Axiom setzen. Das geht ewig so weiter. Wir haben (nach unendlich viel
vergangener Zeit und unendlich vielen neuen Axiomen) also wieder ein System,
das aber nicht mehr algorithmisch lösbar ist, weil Algorithmen, die etwas
lösen sollen, keine unendliche Menge an Axiomen abarbeiten können.

\textsc{Gödel}s Algorithmus bietet die Möglichkeit, jedem sinnvollen und sinnlosen
Satz\mnote{Siehe dazu S. \pageref{grundzeichenzahlen}. Die gewählten Grundzeichen hängen
vom System ab, das man als Grundlage nehmen möchte. \textsc{Gödel} nimmt das System der Peano-Axiome
mit dem System der Principia Mathematica zusammen zur Vereinfachung seines Beweises.}
eine eineindeutige Nummer zuzuordnen, indem er einige Konstanten
definiert, z.\,B. wird \frq 0\flq\ zu 1, \frq f\flq\ zu 3,
\frq$\thicksim$\flq\ zu 5 usw.
Außerdem nutzt er die Eineindeutigkeit von Primfaktorzerlegungen aus.\mnote{Jede
natürliche Zahl ist entweder selbst Prim oder das Produkt aus $n$ Primzahlen. Die Zerlegung
einer Zahl in Primzahlen ist eineindeutig und jede Zahl hat exakt eine ihr zugeordnete
Primfaktorzerlegung. Nimmt man die Position $p$ als $p$te Primzahl hoch einer Zahl, die
ein spezielles Zeichen symbolisiert, so lässt sich diese Zahl jederzeit zurückwandeln
in den ursprünglichen Satz, weil die Primfaktorzerlegung immer eineindeutig bleibt.}
So kann man den (sinnlosen) Satz:
\frq es gilt nicht, dass 4\flq\ folgendermaßen umwandeln:

\begin{enumerate}
	\item \frq es gilt nicht, dass 4\flq
	\item \frq$\thicksim 4$\flq
	\item \frq$\thicksim ffff0$\flq\ (mit der peano'schen Nachfolgerfunktion $f$)\mnote{Siehe dazu § \refsatz{0 N x}}
	\item Nun nehmen wir die Position $p$ im String (angefangen von links mit 1)\mnote{Siehe dazu § \refsatz{Z(n)}}
		und nehmen die $p$te Primzahl als Basis ($p(p)$), die Ziffer für das Zeichen
		als Exponent: $p(1)^5 \times p(2)^3 \times p(3)^3 \times p(3)^3 \times p(4)^3 \times p(5)^1$.
		Wir erhalten:
		$2^5 \times 3^3 \times 5^3 \times 7^3 \times 11^3 \times 13^1 = 640972332000$. Dies
		können wir, da es unendlich viele Primzahlen gibt\mnote{\customlabel{eineindeutigkeit}{\arabic{commentaryNumber}}Die Eineindeutigkeit der Zuordnung von Zahlenwerten
		zu Zeichenketten bedeutet de facto, dass jede mögliche Zeichenkette existiert;
		diese Eineindeutigkeit Zahl $\longleftrightarrow$ Satz ist wichtig, denn in diesen unendlich
		vielen Sätzen wären alle Texte nach \frq mathematischer Grammatik\flq\ entstanden.
		Vgl. dazu die Idee der \frq\citetitle{libraryofbabel}\flq, wo zwar (wahrscheinlich) nicht alle
		Texte gespeichert werden, sie aber im Bereich der mathematischen Hashfunktion,die
		angewandt wird, um die Seiten \frq statisch\flq\ zu erzeugen, dennoch bereits alle
		\frq existieren\flq; hierzu auch eine metaphysische Anmerkung: damit wird
		ein platonischer Realismus der mathematischen Welt vorausgesetzt, in der das Ergebnis jeder
		Gleichung schon \frq existiert\flq, unabhängig davon, ob ausgerechnet oder nicht. In
		einem z.\,B. ultrafinitistischen Denkmodell wäre \textsc{Gödel} gar kein Problem (womit ich nicht
		dafür argumentieren möchte, sondern eher Platonist bin).}, für
		jede beliebig lange Zeichenkette
		bestehend aus den Grundzeichen machen. Die Zahl 640972332000 lässt sich durch
		Primfaktorzerlegung prinzipiell wieder 1:1 in den Ursprungssatz zurückwandeln, so dass gilt:

		$$ 640972332000 \equiv \text{\frq es gilt nicht, dass 4\flq} $$

		Die Zahl $640972332000$ ist damit eineindeutig ebendiesem Satz zugeordnet und mit dem
		gewählten Alphabet keinem anderen ($\longrightarrow$ Eineindeutigkeit).
\end{enumerate}

Der Begriff der Beweisbarkeit\mnote{Vgl. § \refsatz{Bw}}
lässt sich im System der Principia Mathematica selbst
darstellen als die Möglichkeit der Herleitung aus den Axiomen. Nehmen wir die Abkürzungen
von \textsc{Gödel} selbst, dann können wir die Beweisbarkeit eines Satzes folgendermaßen
darstellen: $\text{Bew}\left(x\right)$.\mnote{Vgl. § \refsatz{Bew}}
Wenn wir also einen Satz der Form:

$$
n \vdash \thicksim \text{Bew}\left(m\right)
$$
(wobei $n$ seine gödelisierte Satznummer und \frq zufälligerweise\flq\ $n = m$,
d.\,h. eine lange Zahl, die eineindeutig dem
Satz zugeordnet ist, ist\footnote{Wie \textsc{Gödel} es selbst sagt, wäre es kein
Problem,\mnote{\customlabel{luegeeins}{\arabic{commentaryNumber}}
Kommentar \ref{luegezwei} ist richtig.} diese Zahl für jede Aussage tatsächlich
aufzuschreiben; es wäre nur sehr umständlich,
siehe Fußnote \ref{leichtaufzuschreiben}.}, der rein zufällig auf sich selbst verweist),
dann haben wir einen solchen Satz, der seine eigene Unbeweisbarkeit
behauptet und im System nicht entscheidbar ist. Wäre er beweisbar, wäre er per definitionem
richtig, weil nur richtige Sätze beweisbar sind, aber wäre er beweisbar, wäre er auch
per definitionem unbeweisbar. Seine Richtigkeit lässt sich aber eben durch
metamathematische Überlegungen zeigen.\mnote{Vgl. Kommentar \ref{metamatematischeueberlegungen}.}

\subsection*{\frq\textit{\dots und verwandter Syteme}\flq}

Für solche Sätze sind Systeme wie das der Principia Mathematica prinzipiell
überflüssig,\mnote{Siehe wieder Fußnote \ref{leichtaufzuschreiben}}
weil diese nur auf den logischen Axiomen aufbauen und Abkürzungen für lange Reihen von
Zeichenketten darstellen. Daher sind alle logisch-formalen Systeme betroffen und die
Principia ist nur ein \frq Einstiegspunkt\flq\ zur Vereinfachung der Thesen.

Benötigt zum Erzeugen solcher Sätze sind nur die logischen Schlussregeln und die
potenzielle Möglichkeit, etwas wie natürliche Zahlen herzuleiten, sowie die Möglichkeit,
im System selbst herauszufinden, ob ein Satz ein Beweis eines anderen Satzes ist, ohne dabei
auf ein Metasystem zurückzugreifen.

\subsection*{$\omega$-Widerspruchsfreiheit}

Ein System, in dem die Axiome keine Widersprüchlichkeiten aufweisen, heißt
\frq widerspruchsfrei\flq. Aber nicht jedes widerspruchsfreie System ist auch
$\omega$-widerspruchsfrei. Die $\omega$-Widerspruchsfreiheit ist rigoroser als
die allgemeine Widerspruchsfreiheit, denn in der $\omega$-Widerspruchsfreiheit
gilt -- neben der allgemeinen Widerspruchsfreiheit --, dass auch keine kontraintuitiven
Schlüsse gezogen werden können dürfen (z.\,B. Sätze, die wahr sind, ohne dass
ihre Wahrheit aus den Axiomen herleitbar ist).

Die Axiome der PM sind widerspruchsfrei, aber mit der Methode der \textsc{Gödel}isierung
lassen sich ebensolche Existenzaussagen treffen, die das System $\omega$-widersprüchlich
machen.\mnote{Dazu muss es nur \textit{irgendwie} möglich sein,
aus dem System etwas wie die natürlichen Zahlen und Addition herleiten zu können, was auf fast alle
praktisch-relevanten logisch-widerspruchsfreien Systeme zutrifft. Dazu muss es auch möglich sein,
dass im genannten System der Begriff \frq die Formel $f$ ist beweisbar im
System $S$\flq\ definierbar ist.} Dies gilt für alle ausreichend komplexen
Systeme. Wenn ein System ausreichend komplex ist, dann ist es notwendigerweise unvollständig,
d.\,h. es gibt Existenzaussagen, die nicht aus den Axiomen herleitbar sind (erster
Unvollständigkeitssatz)\mnote{Vgl. Satz VIII auf S. \pageref{satzvii}.}. Kann man jedoch die
Vollständigkeit des ausreichend-komplexen Systems beweisen, ist es notwendigerweise
widersprüchlich (zweiter Unvollständigkeitssatz).\mnote{Vgl. Satz XI auf S. \pageref{satzxi}.}

\subsection*{Vollständigkeit und Entscheidbarkeit}

Ein formales System $P$ ist vollständig, wenn jede Aussage $p \in P$ entweder
exakt \textit{wahr} oder \textit{falsch} ist. Wenn im System für jeden Wert
von $p$ der  Wahrheitswert algorithmisch anhand der Regeln des
formalen Systems bestimmt werden kann, heißt das System \frq entscheidbar\flq.

Es gibt Systeme, auf die die gödelschen Unvollständigkeitssätze nicht zutreffen,\mnote{Ohne
Primzahlen lassen sich die gödelschen Unvollständigkeitssatzbeweise
nicht konstruieren, weil es keine sonstige Möglichkeit gibt, jedem Satz eine eineindeutige,
algorithmisch bestimmbare Nummer zuzuweisen. Vgl. dazu \citetitle{presburger2}.} beispielsweise die Presburger-Arithmetik;
diese sind aber sehr eingeschränkt
(z.\,B. unterstützt die \citefield{presburger}{note} keine Aussagen über Multiplikation,
Division oder Primzahlen).

\subsection*{Die beiden Unvollständigkeitssätze}

Aus der Arbeit \textsc{Gödel}s folgen zwei fundamentale Sätze zur Metamathematik:

\begin{enumerate}
	\item Jedes ausreichend komplexe System beinhaltet immer Sätze, die innerhalb
		des Systems nicht bewiesen werden können.
	\item Jedes ausreichend komplexe System kann seine eigene Widerspruchsfreiheit
		nicht beweisen; wenn es das kann, dann ist es\mnote{Zweiter Unvollständigkeitssatz,
		vgl. Kommentar \ref{zweiterunvollstaendigkeitssatz}.}
		notwendigerweise widersprüchlich.
\end{enumerate}

\subsection*{Auswirkungen}

Die Auswirkungen auf das Verständnis der Mathematik waren gewaltig, denn plötzlich war klar,
dass man nicht jede wahre Aussage auch beweisen kann. Aber auch für die Berechenbarkeitstheorie
und die später aufkommende Informatik sind betroffen. \textsc{Gödel}s Ansatz, mathematische
Aussagen als Ansammlungen von nach Regeln gebildeter Zeichen zu sehen und nicht primär
inhaltlich zu deuten, hat die moderne Informatik überhaupt erst möglich gemacht.
Kein Computer kann mehr als ebendies: Zeichenketten nach genau vorgelegten Regeln
zu manipulieren, ohne dabei auch nur im Ansatz inhaltlich auf diese einzugehen. \textsc{Gödel} hat
gezeigt, dass es immer Aussagen gibt, die ein Computer nicht zurückführen kann auf
seine Axiome, d.\,h. dass es immer Aussagen gibt, die mit einem Computer nicht überprüfbar sind.
Das setzt -- ähnlich wie das Halteproblem -- der Theorie der Algorithmen sowie
der Beweistheorie überhaupt enge Grenzen.

\subsection*{Ziel dieser Arbeit}

Im Rahmen dieser Arbeit würde ich gern versuchen, den Originalalgorithmus soweit
es mir möglich ist nachzuvollziehen. Dazu habe ich die Arbeit von \textsc{Gödel} soweit ich
es konnte exakt abgetippt und mit \LaTeX, einem Textsatzsystem für mathematische
Texte, komplett neu gesetzt, um einige Dinge wie eingefärbte passende Klammern zur besseren
Lesbarkeit einzuführen und Kommentare hinzuzufügen, die die gödel'schen Formeln in eine modernere
Schreibweise übertragen. Trotz größtmöglicher Mühe und Bemühung um Genauigkeit
kann ich dabei natürlich nicht vermeiden, dass sich eventuell Tipp- oder Verständnisfehler
eingeschlichen haben.

\subsection*{Lizenz}

Da es meiner Recherchen nach noch keine solche Version gab, habe ich sie der
Forschung auf meinem GitHub-Account unter der GPL2-Lizenz zur Verfügung
gestellt\footnote{Siehe \citeurl{github}.}.
Jeder kann diese Version forken und eigene Änderungen hinzufügen, um das
Dokument möglichst vollständig zu machen und meine Fehler zu korrigieren.
Es unterliegt keinen Einschränkungen für Forschung und Lehre, außer dass
(wie in der GPL2 mit \textit{Copyleft} üblich) jede Änderung selbst wieder
öffentlich gemacht und unter die GPL2 gestellt werden muss.

Juristisch beziehe ich mich auf §~3 des Urheberrechtsgesetzes, das eine
Bearbeitung und Kommentierung eines Werkes mit eigener Schöpfungshöhe
erlaubt.

\newpage

\printbibliography[keyword={own}]

\newpage
\pagenumbering{arabic}
\setcounter{footnote}{0}
\setcounter{page}{173}
\let\thefootnote\originalthefootnote


\section*{\citefield{goedel}{title}\protect\footnote{Vgl. die im Anzeiger der Akad. d. Wiss. in Wien
(math.-naturw. KI.) 1930, Nr. 19 erschienene Zusammenfassung der Resultate dieser Arbeit.}}

\begin{center}
Von \textbf{Kurt Gödel} in Wien.
\end{center}

\begin{center}
1.
\end{center}

Die Entwicklung der Mathematik in der Richtung zu
größerer\mnote{Das Hilbert-Programm forderte, dass alle mathematischen
Sachverhalte zurückführbar sein müssen auf einfache logische Operationen.
In der Principia Mathematica haben Russell und Whitehead genau das zu erreichen versucht.}
Exaktheit hat bekanntlich dazu geführt, daß weite Gebiete
von ihr formalisiert wurden, in der Art daß das Beweisen
nach einigen wenigen mechanischen Regeln vollzogen werden
kann. Die umfassendsten derzeit aufgestellten formalen
Systeme sind das System der Principia Mathematica
(PM)\footnote{A. Whitehead und B. Russell, Principia
Mathematica, 2. Aufl., Cambridge 1925. Zu den Axiomen
des Systems PM rechnen wir insbesondere auch:
Das Unendlichkeitsaxiom (in der Form: es gibt
genau abzählbar viele Individuen),
das Reduzibilitäts- und das Auswahlaxiom (für alle
Typen).} einerseits, das Zermelo-Fraenkelsche
(von J.~v.~Neumann weiter ausgebildete) Axiomensystem
der Mengenlehre\footnote{Vgl. A. Fraenkel, Zehn
Vorlesungen über die Grundlegung der Mengenlehre,
Wissensch. u. Hyp. Bd. XXXI. J.~v.~Neumann, die
Axiomatisierung der Mengenlehre. Math. Zeitschr. 27,
1928. Journ. f. reine u. angew. Math. 154 (1925),
160 (1929). Wir bemerken, daß man zu den in der
angeführten Literatur gegebenen mengentheoretischen
Axiomen noch die Axiome und die Schlußregeln des
Logikkalküls hinzufügen muß, um die Formalisierung
zu vollenden. -- Die nachfolgenden Überlegungen gelten
auch für die in den letzten Jahren von D. Hilbert und
seinen Mitarbeitern aufgestellten formalen Systeme
(soweit diese bisher vorliegen). Vgl. D. Hilbert,
Math. Ann. 88, Abh. aus d. math. Sem. der Univ.
Hamburg I (1922), VI (1928). P. Bernays, Math. Ann. 93.}
andererseits. Diese beiden Systeme sind so weit, daß
alle heute in der Mathematik angewendeten
Beweismethoden in ihnen formalisiert, d.\,h. auf
einige wenige Axiome und Schlußregeln zurückgeführt sind.
Es liegt daher die Vermutung nahe, daß diese Axiome
und Schlußregeln dazu ausreichen, alle mathematischen
Fragen, die sich in den betreffenden Systemen überhaupt
formal ausdrücken lassen, auch zu entscheiden.\mnote{Das
ist die Grundfrage vom Hilbertprogramm: lassen sich alle mathematischen
Fragestellungen von innerhalb eines formalisierten Systems aus
lösen?} Im folgenden
wird gezeigt, daß dies nicht der Fall ist, sondern daß
es in den beiden angeführten Systemen sogar relativ
einfache Probleme aus der Theorie der gewöhnlichen Zahlen
gibt\footnote{\label{fussnote4}D.\,h. genauer, es gibt unentscheidbare
Sätze, in denen außer den logischen Konstanten:
$\overline{\phantom{XX}}$ (nicht), $\lor$ (oder),
$\left(x\right)$ (für alle), $=$ (identisch mit) keine anderen
Begriffe vorkommen als $+$ (Addition), $.$
(Multiplikation), beide bezogen auf natürliche Zahlen,
wobei auch die Präfixe (x) sich nur auf natürliche Zahlen beziehen dürfen.}, die sich aus den Axiomen nicht
entscheiden lassen. Dieser Umstand liegt nicht etwa\mnote{\customlabel{bedingungen}{\arabic{commentaryNumber}}Alle
Systeme, die komplex genug sind, dass sich aus
ihnen etwas wie natürliche Zahlen und Addition auch nur \so{herleiten} lassen, und in denen selbst definierbar
ist, dass eine Aussage ein Beweis einer anderen Aussage ist, sind betroffen}
an der speziellen Natur der aufgestellten Systeme,
sondern gilt für eine sehr weite Klasse formaler Systeme,
zu denen insbesondere alle gehören, die aus den beiden
angeführten durch Hinzufügung endlich vieler Axiome
\mnote{Durch das Hinzufügen von \textit{unendlich} vielen Axiomen wäre das System praktisch nicht mehr verwendbar und es gäbe potenziell gar keine unbewiesenen Aussagen. Daher der Fokus auf die Systeme mit endlich vielen Axiomen.}
entstehen\footnote{Dabei werden in PM nur solche Axiome
als verschieden gezählt, die aus einander nicht bloß
durch Typenwechsel entstehen.}, vorausgesetzt, daß
durch die hinzugefügten Axiome keine falschen Sätze
von denen in der Fußnote \ref{fussnote4} angegebenen
Art beweisbar werden.

Wir skizzieren, bevor wir auf Details eingehen, zunächst
den Hauptgedanken des Beweises, natürlich ohne auf
Exaktheit Anspruch zu erheben. Die Formeln eines formalen Systems (wir beschränken uns hier auf das System PM) sind
äußerlich betrachtet endliche Reihen der Grundzeichen\mnote{Es geht erstmal um eine rein syntaktische Betrachtung, ohne
auf den Inhalt Bezug zu nehmen}
(Variable, logische Konstante und Klammern bzw.
Trennungspunkte) und man kann leicht genau präzisieren,
welche Reihen von Grundzeichen sinnvolle Formeln sind
und welche nicht\footnote{Wir verstehen hier und im folgenden
Unter \glqq Formel aus PM\grqq\ immer eine ohne Abkürzungen
(d.\,h. ohne Verwendung von Definitionen)
geschriebene Formel. Definitionen dienen ja nur der kürzeren
Schreibweise und sind daher prinzipiell überflüssig.}.
Analog sind Beweise vom formalen Standpunkt nichts
anderes\mnote{\customlabel{tractatus}{\arabic{commentaryNumber}}Ein Beweis ist eine für den Computer nachvollziehbare
Kette an a., entweder Axiomen oder b., Schlussfolgerungen aus den
Axiomen (vgl. \protect\citetitle{wittgenstein}, § 6.126 und § 6.1262)}
als endliche Reihen von Formeln (mit bestimmten angebbaren
Eigenschaften). Für metamathematische Betrachtungen
ist es natürlich gleichgültig, welche Gegenstände man
als Grundzeichen nimmt, und wir entschließen uns dazu,\mnote{Natürliche Zahlen sind relativ einfach über die
Peano-Axiome herleitbar und zu verstehen. Es könnten aber auch andere mathematische Strukturen genommen werden,
wenn sie aus den Axiomen folgen, die dann analoge Schlüsse zulassen.}
natürliche Zahlen\footnote{D.\,h. wir bilden die Grundzeichen
in eineindeutiger Weise auf die natürlichen Zahlen ab (Vgl. die Durchführung auf S. \pageref{grundzeichenzahlen}.)}
als solche zu verwenden. Dementsprechend ist dann eine
Formel eine endliche Folge natürlicher\mnote{Vgl. Kommentar \ref{eineindeutigkeit}}
Zahlen\footnote{D.\,h. eine Belegung eines Abschnittes der
Zahlenreihe mit natürlichen Zahlen. (Zahlen können ja nicht in
räumliche Anordnung gebracht werden.)} und eine Beweisfigur eine
endliche Folge von endlichen Folgen natürlicher Zahlen.
Die metamathematischen Begriffe (Sätze) werden dadurch zu
Begriffen (Sätzen) über natürliche Zahlen bzw. Folgen von
Solchen\footnote{m.\,a.\,W.: Das oben beschriebene Verfahren
liefert ein isomorphes Bild des Systems PM im Bereich der
Arithmetik und man kann alle metamathematischen Überlegungen
ebenso gut an diesem isomorphen Bild vornehmen. Dies geschieht
in der folgenden Beweisskizze, d.\,h. unter \glqq Formel\grqq,
\glqq Satz\grqq, \glqq Variable\grqq\ etc.
\so{sind immer die entsprechenden Gegenstände des isomorphen Bildes zu verstehen}.}
und daher (wenigstens teilweise) in den Symbolen
des Systems PM selbst ausdrückbar. Insbesondere kann man
zeigen, daß die Begriffe \mnote{Vgl. § \refsatz{Logische Operatoren}, \refsatz{Bw}, \refsatz{Bew}}\glqq Formel\grqq,
\glqq Beweisfigur\grqq,
\glqq beweisbare Formel\grqq\ innerhalb des Systems PM
definierbar sind, d.\,h. man kann z.\,B. eine Formel
$F\left(v\right)$ aus PM mit einer freien Variablen $v$ (vom Typus
einer Zahlenfolge) angeben\footnote{\label{leichtaufzuschreiben}Es wäre sehr leicht
(nur etwas umständlich), diese Formel tatsächlich hinzuschreiben.},
so daß $F\left(v\right)$ inhaltlich interpretiert besagt:
$v$ ist eine beweisbare Formel.\mnote{Das ist eine rein-syntaktische Definition, die zahlentheoretisch
aus den Axiomen überprüft werden kann (zumindest im Beispielsystem der PM). Vgl. § \refsatz{Bew}}
Nun stellen wir einen
unentscheidbaren Satz des Systems PM, d.\,h. einen
Satz $A$, für den weder $A$ noch \textit{non}-$A$ beweisbar
ist, folgendermaßen her:

Eine Formel aus PM mit genau einer freien Variable, u. zw.
vom Typus der natürlichen Zahlen (Klasse von Klassen)
wollen wir ein \so{Klassenzeichen} nennen. Die
Klassenzeichen denken wir uns irgendwie in eine Folge
geordnet\footnote{Etwa nach steigender Gliedersumme und
bei gleicher Summe lexikographisch.},
bezeichnen das $n$-te mit $R\left(n\right)$ und merken, daß sich
der Begriff \glqq Klassenzeichen\grqq\ sowie die ordnende
Relation $R$ im System PM definieren lassen. Sei
$\alpha$ ein beliebiges Klassenzeichen, mit $\left[\alpha; n\right]$
bezeichnen wir diejenige Formel, welche aus dem
Klassenzeichen $\alpha$ dadurch entsteht, daß man die
freie Variable durch das Zeichen für die natürliche Zahl\mnote{Da jede Zahl eineindeutig einer Formel zugeordnet
werden kann (durch Primfaktorzerlegung), gibt es in $\mathbb{N}$ alle sinnvollen und sinnlosen Formeln,
die denkbar sind. Das heißt, dort gibt es alle beweisbaren Formeln, aber auch alle die, die nicht beweisbar sind. Gödel
definiert hier ein $n \in K$ so, dass $n \leftrightarrow$ eine nicht beweisbare Formel ist.}
$n$ ersetzt. Auch die Tripel-Relation
$\alpha = \left[y; z\right]$ erweist sich als innerhalb von PM
definierbar. Nun definieren wir eine Klasse $K$ natürlicher
Zahlen folgendermaßen:

\let\originalfootnote=\thefootnote
\let\thefootnote=\fnelfa
\mnote{$\exists n: n \in K \wedge \thicksim Bew \left[R\left(n\right); n\right]$, vgl. § \refsatz{Bew}}
\begin{equation}
\label{formel1}
n \epsilon K \equiv \overline{Bew}\left[R\left(n\right); n\right]\footnote{Durch Überstreichen wird die Negation bezeichnet.}
\end{equation}
\let\thefootnote=\originalfootnote
\setcounter{footnote}{11}

(wobei $Bew\ x$ bedeutet: $x$ ist eine beweisbare Formel\mnote{Vgl. § \refsatz{Bew}}).
Da die Begriffe, welche im Definiens vorkommen, sämtlich
in PM definierbar sind, so auch der daraus zusammengesetzte
Begriff $K$, d.\,h. es gibt ein Klassenzeichen
$S$\footnote{Es macht wieder nicht die geringsten Schwierigkeiten, die Formel $S$ tatsächlich hinzuschreiben.},
so daß die Formel $\left[S; n\right]$ inhaltlich gedeutet besagt,
daß die natürliche Zahl $n$ zu $K$ gehört.\mnote{$n \in K$} $S$ ist als
Klassenzeichen mit einem bestimmten $R\left(q\right)$ identisch,
d.\,h. es gilt

$$ S = R\left(q\right) $$%

\noindent für eine bestimme natürliche Zahl $q$. Wir zeigen nun, daß
der Satz $\left[R\left(q\right); q\right]$\footnote{Man beachte, daß
\glqq$\left[R\left(q\right); q\right]$\grqq\ (oder was dasselbe bedeutet
\glqq$\left[S; q\right]$\grqq) bloß eine \so{metamathematische Beschreibung}
des unentscheidbaren Satzes ist. Doch kann man, sobald man die Formel
$S$ ermittelt hat, natürlich auch die Zahl $q$ bestimmen und damit
den unentscheidbaren Satz selbst effektiv hinschreiben.}
in PM unentscheidbar ist. Denn angenommen, der Satz\mnote{Wäre der Satz beweisbar, wäre er per definitionem wahr und damit unbeweisbar. Ein logischer Widerspruch.}
$\left[R\left(q\right); q\right]$ wäre beweisbar, dann wäre er auch richtig,
d.\,h. aber nach dem obigen $q$ würde zu $K$ gehören,
d.\,h. nach \ref{formel1} es würde
$\overline{Bew}\left[R\left(q\right); q\right]$ gelten, im Widerspruch mit der
Annahme. Wäre dagegen die Negation von
$\left[R\left(q\right); q\right]$ beweisbar, so würde $n \epsilon K$,
d.\,h. $Bew\left[R\left(q\right); q\right]$ gelten. $\left[R\left(q\right); q\right]$ wäre also
zugleich mit seiner Negation beweisbar, was wiederum
unmöglich ist.

Die Analogie dieses Schlusses mit der Antinomie Richard\mnote{Der Satz bezieht sich nicht direkt auf sich selbst,
sondern nur \glqq zufällig\grqq, weil er eine von unendlich vielen Möglichkeiten in $\mathbb{N}$ ist, von denen
einige sich auf sich selbst beziehen müssen. Damit ist die Typentheorie Russells \frq ausgehebelt\flq, die nur
explizite Eigenbezüge verbieten kann, aber keine rein-zufälligen.}
springt in die Augen. Auch mit dem
\glqq Lügner\grqq\ besteht eine nahe
Verwandtschaft\footnote{Es läßt sich überhaupt jede epistemologische
Antinomie zu einem derartigen Unentscheidbarkeitsbeweis verwenden.},
denn der unentscheidbare Satz $\left[R\left(q\right); q\right]$ besagt ja,
daß $q$ zu $K$ gehört, d.\,h. nach \ref{formel1},
daß $\left[R\left(q\right); q\right]$ nicht beweisbar ist. Wir haben
also einen Satz vor uns, der seine eigene
Unbeweisbarkeit behauptet\footnote{\label{zufaellig}Ein solcher Satz
hat entgegen dem Anschein nichts Zirkelhaftes an sich, denn
er behauptet zunächst die Unbeweisbarkeit einer ganz bestimmenten
Formel (nämlich der $q$-ten in der lexikographischen Anordnung
bei der bestimmten Einsetzung), und erst nachträglich (gewissermaßen
zufälligen) stellt sich heraus, daß diese Formel gerade
die ist, in der er selbst ausgedrückt wurde.}.

Die eben auseinandergesetzte Beweismethode läßt sich\mnote{Damit sind extrem viele Systeme betroffen, denn
in fast allen komplexeren Systemen sind die Mittel dafür irgendwie zumindest prinzipiell herleitbar; das reicht.}
offenbar auf jedes formale System anwenden, das erstens
inhaltlich gedeutet über genügen Ausdrucksmittel
verfügt, um die in der obigen Überlegung vorkommenden
Begriffe (insbesondere den Begriff \glqq beweisbare\mnote{Die Bedingungen für die Unvollständigkeit: der Begriff
\frq beweisbare Formel\flq\ muss im System definiert sein, jede beweisbare Formel muss inhaltlich richtig sein.}
Formel\grqq) zu definieren, und in dem zweitens jede
beweisbare Formel auch inhaltlich richtig ist. Die
nun folgende exakte Durchführung des obigen Beweises
wird unter anderem die Aufgabe haben, die zweite der eben
angeführten Voraussetzungen durch eine rein formale und
weit schwächere zu ersetzen.

Aus der Bemerkung, daß $\left[R\left(q\right); q\right]$ seine
eigene\mnote{\customlabel{metamatematischeueberlegungen}{\arabic{commentaryNumber}} Wenn der Satz
$S = \left[R\left(q\right); q\right]$ unbeweisbar ist und er nicht bewiesen werden kann,
dann ist er wahr. Das kann dann durch die metamathematische Überlegung gezeigt werden,
dass man eben sieht, dass die Aussage wahr ist, weil sie eben nicht beweisbar ist.}
Unbeweisbarkeit behauptet, folgt sofort, daß $\left[R\left(q\right); q\right]$
richtig ist, denn $\left[R\left(q\right); q\right]$ ist ja unbeweisbar (weil
unentscheidbar). Der im System PM unentscheidbare Satz
wurde also durch metamathematische Überlegungen doch
entschieden. Die genaue Analyse dieses merkwürdigen
Umstandes führt zu überraschenden Resultaten, bezüglich der
Widerspruchsfreiheitsbeweise formaler System, die in
Abschn. 4. (Satz XI) näher behandelt werden.

\begin{center}
2.
\end{center}

Wir gehen nun an die exakte Durchführung des
oben skizzierten Beweises und geben zunächst eine
genaue Beschreibung des normalen Systems $P$, für welches\mnote{$P = \text{PM} + \text{Peano-Axiome}$}
für die Existenz unentscheidbarer Sätze nachweisen wollen.
$P$ ist im wesentlichen das System, welches man erhält,
wenn man die Peanoschen Axiome mit der Logik
der PM\footnote{Die Hinzufügung der Peanoschen
Axiome ebenso wie alle anderen am System PM angebrachten
Abänderungen dienen lediglich zur Vereinfachung des
Beweises und sind prinzipiell entbehrlich.}
überbaut (Zahlen als Individuen, Nachfolgerelation
als undefinierten Grundbegriff).

\label{ersetzungdurchklassen}
Die Grundzeichen des Systems $P$ sind die folgenden:\mnote{Wieder erstmal rein syntaktisch}

\begin{enumerate}[I.]
	\item Konstante: \glqq $\thicksim$\grqq\ (nicht),\mnote{Siehe die auf S. \pageref{grundzeichenzahlen} definierten Grundzeichen, die je konstant einer Zahl zugeordnet sind.}
		\glqq$\lor$\grqq\ (oder), \glqq$\Pi{}$\grqq\ (für alle),
		\glqq $0$\grqq\ (Null), \glqq$f$\grqq\ (der Nachfolger
		von), \glqq(\grqq, \glqq)\grqq\ (Klammern),

	\item Variable ersten Typs (für Individuen, d.\,h. \mnote{Für Individuenvariablen.}
		natürliche Zahlen inklusive 0):
		\glqq$x_1$\grqq, \glqq$y_1$\grqq,\glqq$z_1$\grqq, \dots,

		Variable zweiten Typs (für Klassen von Individuen):\mnote{Für Klassen, die aus einzelnen Individuenvariablen bestehen.}
		\glqq$x_2$\grqq, \glqq$y_2$\grqq, \glqq$z_3$\grqq, \dots,

		Variable dritten Typs (für Klassen von Klassen von\mnote{Für Klassen, deren Elemente selbst wieder Klassen sind.}
		Individuen): \glqq$x_3$\grqq, \glqq$y_3$\grqq,
		\glqq$z_3$\grqq, \dots usw. für jede natürliche Zahl
		als Typus\footnote{Es wird vorausgesetzt, daß für
		jeden Variablentypus abzählbar viele Zeichen zur Verfügung stehen.}.
\end{enumerate}

Anm.: Variable für zwei- und mehrstellige Funktionen\mnote{Somit wird $f(x, y,  z, \hdots)$ zu $f(\mathfrak{x})$ und $\mathfrak{x}$ expandiert zu
$(x, y, z, \dots)$, vgl. Fußnote \ref{fnfraktur}.}
(Relationen) sind als Grundzeichen überflüssig, da man
Relationen als Klassen geordneter Paare definieren kann
und geordnete Paare wiederum als Klassen von Klasse, z.\,B.
das geordnete Paar $a, b$ durch $\left(\left(a\right),
\left(a, b\right)\right)$, wo $\left(x, y\right)$ bzw. $\left(x\right)$ die
Klassen bedeuten, deren einzige Elemente $x, y$ bzw. $x$
sind\footnote{Auch inhomogene Relationen können auf diese
Weise definieren werden, z.\,B. eine Relation zwischen
Individuen und Klassen als eine Klasse aus Elementen
der Form: $\left(\left(x_2\right), \left(\left(x_1\right), x_3\right)\right)$.
Alle in den PM über Relationen beweisbaren Sätze sind, wie eine
einfache Überlegung lehrt, auch bei dieser Behandlungsweise beweisbar.}.

Unter einem Zeichen ersten Typs verstehen wir eine Zeichenkombination der Form:\mnote{Damit sind
mit der peanoschen Nachfolgerfunktion gemeint: $0, 1, 2, 3, \dots$. Daher auch der Name
\frq Zahlzeichen\flq.}

$$ a, fa, ffa, fffa \dots \text{usw.} $$
wo $a$ entweder 0 oder eine Variable ersten Typs ist.
Im ersten Fall nennen wir ein solches Zeichen Zahlzeichen.
Für $n > 1$ verstehen wir unter einem Zeichen $n$-ten
Typs dasselbe wie Variable $n$-ten Typs.
Zeichenkombinationen der Form $a\left(b\right)$, wobei
$b$ ein Zeichen $n$-ten und $a$ ein Zeichen $n + 1$-ten
Typs ist, nennen wir Elementarformeln\mnote{\customlabel{elementarformeln}{\arabic{commentaryNumber}}Eine \frq Elementarformel\flq\ ist
eine Zuordnung von Mengen zu Zahlen (oder Mengen von Mengen zu Mengen).}. Die Klasse der
Formeln definieren wir als die kleinste
Klasse\footnote{Bez. dieser Definition (und analoger
später vorkommender), vgl. J. \L{}ukasiewicz und
A. Tarski, Untersuchungen über den Aussagenkalkül,
Comptes Rendus des s\'eances de la Soci\'et\'e de
Sciences et Lettres de Varsovie XXIII, 1930, Cl. III.},
zu welcher sämtliche Elementarformeln gehören und zu\mnote{\customlabel{minimalmenge}{\arabic{commentaryNumber}}Damit ist die Minimalmenge \frq sinnvoller\flq\ Sätze
gemeint, die alle Formeln
beinhaltet, d.\,h. $a, b; \thicksim a; a \lor b; , \forall x: (a)$ mit $x$ als beliebigem Zahlzeichen.
Praktisch ist das dann ein großes Array von langen Zahlen, deren Verneinungen, deren Adjunktion und deren
Generalisierung(auch alles Zahlen).}
welcher zugleich mit $a, b$ stets auch
$\thicksim\left(a\right)$, $\left(a\right)\lor\left(b\right)$, $x\Pi\left(a\right)$ gehören (wobei $x$ eine
\let\originalfootnote=\thefootnote
\let\thefootnote=\fnachtzehna
beliebige Variable ist)\footnote{$x\Pi\left(a\right)$ ist also auch dann eine
Formel, wenn $x$ in $a$ nicht oder nicht frei vorkommt. In diesem
Fall bedeutet $x\Pi\left(a\right)$ natürlich dasselbe wie $a$.}
\let\thefootnote=\originalfootnote
\setcounter{footnote}{19}
$\left(a\right)\lor\left(b\right)$. nennen wir Disjunktion aus $a$ und $b$,
$\thicksim\left(a\right)$ die Negation und $x\Pi\left(a\right)$ eine Generalisation von
$a$. Satzformel heißt eine Formel, in der keine freie Variable
vorkommt (freie Variable in der bekannten Weise
definiert). Eine Formel mit genau $n$-freien
Individuenvariablen (und sonst keinen freien Variablen)
nennen wir $n$-stelliges Relationszeichen, für $n = 1$
auch Klassenzeichen.

Unter $\text{Subst\ } a \Spvek{v; b}$ (wo $a$ eine Formel,\mnote{$\text{Subst\ } a \Spvek{v; b} = $
\mcode{\$a =\~{} s/\textbackslash bv\textbackslash b/b/g}}
$v$ eine Variable und $b$ ein Zeichen vom selben Typ wie
$v$ bedeutet) verstehen wir die Formel, welche aus $a$
entsteht, wenn man darin $v$ überall, wo es frei ist, durch
$b$ ersetzt\footnote{Falls $v$ in $a$ nicht als freie
Variable vorkommt, soll $\text{Subst\ } a \Spvek{v; b} = a$
sein. Man beachte, daß \glqq Subst\grqq\ ein Zeichen der Metamathematik ist.}.
Wir sagen, daß eine Formel $a$ eine Typenerhöhung einer\mnote{$a$ ist Typenerhöhung von $b$, wenn
man $a$ aus $b$ konstruieren kann, indem man die Typen (Zahlzeichen oder Klasse oder Klasse
von Klassen) aller in $b$ vorkommenden Variablen erhöht.}
anderen $b$ ist, wenn $a$ aus $b$ dadurch entsteht, daß
man den Typus aller in $b$ vorkommenden Variablen um die
gleiche Zahl erhöht.

Folgende Formeln (I bis V) heißen Axiome (sie
sind mit Hilfe der in bekannter Weise definierten Abkürzungen:
$., \supset, \equiv, \left(Ex\right), =$\footnote{$x_1 = y_1$ ist,
wie in PM I, $\pmstar$ 13 durch
$x_2\Pi\left(x_2\left(x_1\right) \supset x_2\left(y_1\right)\right)$
definiert zu denken (ebenso für die höheren Typen).}
und mit der Verwendung der üblichen Konventionen über
das Weglassen von Klammern geschrieben\footnote{Um aus den
angeschriebenen Schemata die Axiome zu erhalten,
muß man also (in II, II, IV nach Ausführung der erlaubten Einsetzungen) noch
\begin{enumerate}
\item die Abkürzungen elimieren,
\item die unterdrückten Klammern hinzufügen.
\end{enumerate}

Man beachte, daß die so entstehenden Ausdrücke \glqq Formeln\grqq\ im obigen Sinn sein müssen. (Vgl. auch die
exakten Definitionen der metamathe. Begriffe S. \pageref{genaueformeln}fg.)}):

\begin{enumerate}[I.]
	\item \begin{enumerate}[1.]
		\item $\thicksim\left(fx_1 = 0\right)$\mnote{0 ist nicht Nachfolger irgendeiner natürlichen Zahl}
		\item $fx_1 = fy_1 \supset x_y = y_1$\mnote{Wenn die Nachfolger gleich sind, ist die Zahl gleich}
		\item $x_2\left(0\right).x_1\Pi\left(x_2\left(x_1\right) \supset x_2\left(fx_1\right)\right) \supset x_1 \Pi \left(x_2\left(x_1\right)\right)$.
	\end{enumerate}
	\item Jede Formel, die aus den folgenden Schemata
	durch Einsetzung beliebiger Formeln für $p$, $q$,
	$r$ entsteht.
	\begin{enumerate}[1.]
		\item $p\lor p\supset p$\mnote{Aus $p$ oder $p$ folgt $p$}
		\item $p \supset p \lor p$\mnote{Aus $p$ folgt $p$ oder $p$}
		\item $p \lor q \supset q \lor p$\mnote{$p$ und $q$ sind im ODER vertauschbar}
		\item $\left(p \supset q\right) \supset \left(r \lor p \supset r \lor q\right)$\mnote{Wenn aus $p$ $q$ folgt, kann man an beide ein beliebiges $r$ ran-ODERN und der Wahrheitswert bleibt gleich}
	\end{enumerate}

	\item Jede Formel, die aus einem der beiden Schemata\mnote{Formeln dieser Art werden
		der \frq Einfachheit\flq\ halber als Axiome gesetzt.}

	\begin{enumerate}[1.]
		\item $v\Pi\left(a\right) \supset \text{Subst\ } a \Spvek{v; c}$\mnote{$\forall v: a \supset \text{Subst } \Spvek{v ;c }$; $v$ Variable, $c$ selber Typus wie $v$}
		\item $v\Pi\left(b \lor a\right)\supset b \lor v \Pi\left(a\right)$\mnote{$\forall v: (b \lor a) \supset b \lor \forall v: a$}
	\end{enumerate}
	dadurch entsteht, daß man für $a, v, b, c$ folgende
	Einsetzungen vornimmt (und in 1. die durch \glqq Subst\grqq\ angezeigte Operation ausführt):

	Für $a$ eine beliebige Formel, für $v$ eine beliebige Variable, für $b$ eine Formel,
	in der $v$ nicht frei vorkommt, für $c$ ein Zeichen vom selben Typ wie $v$,
	vorausgesetzt, daß $c$ keine Variable enthält, welche in $a$ an einer Stelle
	gebunden ist, an der $v$ frei ist\footnote{$c$ ist also entweder eine Variable
	oder 0 oder ein Zeichen der Form $f \dots fu$, wo $u$ entweder 0 oder eine
	Variable 1. Typs ist. Bez. des Begriffs \glqq frei (gebunden) an einer Stelle
	von $a$\grqq, vgl. die Fußnote \ref{fussnote24} zitierte Arbeit I A 5.}.

	\item Jede Formel, die aus dem Schema

	\begin{enumerate}[1.]
		\item $\left(Eu\right)\left(v\Pi\left(u\left(v\right) \equiv a\right)\right)$\mnote{$\exists u: \forall v: u\left(v\right) \equiv a$}
	\end{enumerate}

	dadurch entsteht, daß man für $v$ bzw. $u$
	beliebige Variablen  vom Typ $n$
	bzw. $n + 1$ und für $a$ eine Formel, die $u$
	nicht frei enthält, einsetzt. Dieses Axiom
	vertritt das Reduzibilitätsaxiom
	(Komprehensionsaxiom der Mengenlehre).

	\item Jede Formel, die aus der folgenden durch
		Typenerhöhung entsteht (und diese Formel selbst):\mnote{$\forall x_y: \left(x_2\left(x_1\right) \equiv y_2\left(x_1\right)\right) \supset x_2 = y_2$. Hier wird die Äquivalenz von zwei Mengen darüber definiert, dass all ihre Elemente gleich sind.}

	\begin{enumerate}[1.]
		\item $x_1\Pi\left(x_2\left(x_1\right)\equiv y_2\left(x_1\right)\right) \supset x_2 = y_2$.
	\end{enumerate}

	Dieses Axiom besagt, daß eine Klasse durch
	ihre Elemente vollständig bestimmt ist.
\end{enumerate}

Eine Formel $c$ heißt unmittelbare Folge aus $a$ und $b$
(bzw. aus $a$), wenn $a$ die Formel $\left(\thicksim\left(b\right)\right)\lor\left(c\right)$
ist (bzw. wenn $c$ die Formel $v\Pi\left(a\right)$\mnote{$\forall v: a$} ist,
wo $v$ eine beliebige Variable bedeutet). Die Klasse
der beweisbaren Formeln wird definiert als die kleinste
Klasse von Formeln, welche die Axiome enthält und gegen
die Relation \glqq unmittelbare Folge\grqq\ abgeschlossen
ist\footnote{\label{fussnote24}Die Einsetzungsregel wird
dadurch überflüssig, daß wir alle möglichen Einsetzungen
bereits in den Axiomen selbst vorgenommen haben (analog bei
J.~v.~Neumann, Zur \so{Hilbert}schen Beweistheorie, Math. Zeitschr. 26, 1927.)}.

Wir ordnen nun den Grundzeichen des Systems $P$ in\mnote{Es geht um eine rein syntaktische
Betrachtung zum Umwandeln von Formeln in natürliche Zahlen. Gödel bietet einen generalisierten
Algorithmus an, um in jedem ausreichend komplexen System Sätze zu formulieren, die nicht nicht-beweisbar
sind. Dazu nutzt er zur Vereinfachung die Peano-Axiome, die aber, da sie selbst Folgen der logischen
Annahmen sind, an sich entbehrlich sind.} folgender Weise eineindeutig natürliche Zahlen zu:

\label{grundzeichenzahlen}
\begin{center}
\begin{tabular}{lll}
	\glqq$0$\grqq\dots 1 & \glqq$\lor$\grqq\dots 7 & \glqq(\grqq\dots 11 \\
	\glqq$f$\grqq\dots 3 & \glqq$\Pi$\grqq\dots 9 & \glqq)\grqq\dots 13\\
	\glqq$\thicksim$\grqq\dots 5 & &
\end{tabular}
\end{center}

\noindent ferner den Variablen $n$-ten Typs die Zahlen der
Form $p^n$ (wo $p$ eine Primzahl $> 13$ ist). Dadurch
entspricht jeder endlichen Reihe von Grundzeichen (also\mnote{Gödelisierung: Zahl $\leftrightarrow$ Formel}
auch jeder Formel) in eineindeutiger Weise
eine endliche Reihe natürlicher Zahlen. Die endlichen
Reihen natürlicher Zahlen bilden wir nun (wieder eineindeutig)
auf natürliche Zahlen ab, indem wir der Reihe
$n_1, n_2, \dots n_k$ die Zahl
$2^{n_1}, 3^{n_2}, \dots p_k^{nk}$\mnote{$\text{\glqq}f\left(0\right)\text{\grqq} = 2^1 + 3^{11} + 5^{1} + 7^{13} = 96\,889\,187\,561$,
96\,889\,187\,561 ist eineindeutig zugeordnet zu \glqq$f\left(0\right)$\grqq}
entsprechen lassen, wo $p_k$ die $k$-te
Primzahl (der Größe nach) bedeutet. Dadurch ist nicht nur
jeder endlichen Reihe von solchen in eineindeutiger Weise
eine natürliche Zahl zugeordnet. Die Grundzeichen (bzw.
der Grundzeichenreihe) $a$ zugeordnete Zahl bezeichnen
wir mit $\Phi\left(a\right)$. Sei nun irgend eine Klasse oder
Relation $R\left(a_1, a_2, \dots a_n\right)$ zwischen Grundzeichen
oder Reihen von solchen gegeben. Wir ordnen ihr diejenige
Klasse (Relation) $R'\left(x_1, x_2, \dots x_n\right)$ zwischen
natürlichen Zahlen zu, welche dann und nur dann zwischen
$x_1, x_2, \dots x_n$ besteht, wenn es solche
$a_1, a_2, \dots a_n$ gibt, daß $x_i = \Phi\left(a_i\right) \left(i = 1, 2, \dots n\right)$ und
$R\left(a_1, a_2, \dots a_n\right)$ gilt. Diejenigen
Klassen und Relationen natürlicher Zahlen, welche
auf dieser Weise den bisher definierten
metamathematischen Begriffen, z.\,B. \glqq Variable\grqq,
\glqq Formel\grqq, \glqq Satzformel\grqq, \glqq Axiom\grqq,
\glqq beweisbare Formel\grqq\ usw. zugeordnet sind,
bezeichnen wir mit denselben Worten in Kursivschrift. Der\mnote{Es existiert eine Menge an Aussagen mit
der Mindestkardinalität $1$, so dass die Einzelaussagen nicht beweisbar sind, aber auch das Gegenteil
nicht gezeigt werden kann.}
Satz, daß es im System $P$ unentscheidbare Probleme gibt,
lautet z.\,B. folgendermaßen: es gibt \textit{Satzformeln} $a$,
so daß weder $a$ noch die \textit{Negation} von $a$
\textit{beweisbare Formeln} sind.

\label{zwischenbetrachtungrekursion}
Wir schalten nun eine Zwischenbetrachtung ein, die mit dem
formalen System $P$ vorderhand nichts zu tun hat, und
geben zunächst folgende Definition: eine zahlentheoretische
Funktion\footnote{D.\,h. ihr Definitionsbereich ist die
Klasse der nicht negativen ganzen Zahlen (bzw. der $n$-tupel
von solchen) und ihre Werte sind nicht negative ganze Zahlen.}
$\phi\left(x_1, x_2, \dots x_n\right)$ heißt \so{rekursiv definiert}
aus den zahlentheoretischen Funktionen
$\psi\left(x_1, x_2 \dots x_{n - 1}\right)$ und $\mu\left(x_1, x_2 \dots x_{n + 1}\right)$, wenn
für alle $x_2 \dots x_n$, $k$\footnote{Kleine lateinische Buchstaben (ev.
mit Indizes) sind im folgenden immer Variable für nicht negative
ganze Zahlen (falls nicht ausdrücklich das Gegenteil bemerkt ist).}
folgendes gilt:\mnote{Eine Funktion ist rekursiv, wenn sie intern dargestellt
werden kann als Aufrufe anderer Funktionen (\glqq Wrapper\grqq\ in der Informatik).}

\begin{equation}
	\begin{aligned}
		\phi\left(0, x_2 \dots x_n\right) =  \psi\left(x_2 \dots x_n\right)\\
		\phi\left(k + 1, x_2 \dots x_n\right) = \mu\left(k, \phi\left(k, x_2 \dots x_n\right), x_2 \dots x_n\right)
	\end{aligned}
\end{equation}

Eine zahlentheoretische Funktion $\phi$ heißt
\so{rekursiv}, wenn es eine endliche Reihe von zahlentheor.
Funktionen $\phi_1, \phi_2 \dots \phi_n$ gibt, welche
mit $\phi$ endet und die Eigenschaft hat, daß jede
Funktion $\phi_k$ der Reihe entweder aus zwei
vorhergehenden rekursiv definiert ist oder aus irgend
welchen der vorhergehenden durch Einsetzung
entsteht\footnote{Genauer: durch Einsetzung gewisser der vorhergehenden
Funktionen an die Leerstellen einer der vorhergehenden, z.\,B. $\phi_k\left(x_1, x_2\right) = \phi_p\left[\phi_q\left(x_1, x_2\right), \phi_r\left(x_2\right)\right]$ ($p, q, r < k$). Nicht alle Variable der linken Seite müssen auch rechts vorkommen (ebenso im Rekursionsschma (2)).}
oder schließlich eine Konstante oder die Nachfolgerfunktion
$x + 1$ ist. Die Länge der kürzesten Reihe von $\phi_i$,\mnote{Die minimale \glqq Call Depth\grqq\ der
Funktion ist ihre Stufe}
welche zu einer rekursiven Funktion $\phi$ gehört,
heißt ihre Stufe. Eine Relation zwischen natürlichen
Zahlen $R\left(x_1 \dots x_n\right)$ heißt
rekursiv\footnote{Klassen rechnen wir mit zu den
Relation (einstellige Relationen). Rekursive Relation
$R$ haben natürlich die Eigenschaft, daß man für
jedes spezielle Zahlen-$n$-tupel entscheiden kann,
ob $R\left(x_1 \dots x_n\right)$ gilt oder nicht.},
wenn es eine rekursive Funktion $\phi\left(x_1 \dots x_n\right)$ gibt,
so daß für alle $x_1, x2 \dots x_n$

$$ R\left(x_1 \dots x_n\right) \thicksim \left[\phi\left(x_1 \dots x_n\right) = 0\right]\footnote{Für alle inhaltlichen (insbes. auch die metamathematischen) Überlegungen wird die \so{Hilbert}sche Symbolik verwendet. Vgl. Hilbert-Ackermann, Grundzüge der
theoretischen Logik, Berlin 1928.}$$

Es gelten folgende Sätze:

\begin{enumerate}[I.]
	\item \so{Jede aus rekursiven Funktionen (Relationen) durch Einsetzung
		rekursiver Funktionen an Stelle der Variablen entstehende Funktion (Relation)
		ist rekursiv; ebenso jede Funktion, die aus rekursiven Funktionen durch rekursive
		Definition nach dem Schema (2) entsteht.}
	\item \so{Wenn $R$ und $S$ rekursive Relationen sind,
	dann auch $R$, $R\lor S$ (daher auch $R \& S$)}.
	\item \so{Wenn die Funktionen $\phi\left(\mathfrak{x}\right), \psi\left(\mathfrak{y}\right)$
		rekursiv sind, dann auch die Relation
	$\phi\left(\mathfrak{x}\right) = \psi\left(\mathfrak{y}\right)$\footnote{\label{fnfraktur}
		Wir verwenden deutsche
		Buchstaben $\mathfrak{x}, \mathfrak{y}$ als abkürzende Bezeichnung für beliebige
		Variablen-$n$-tupel, z.\,B. $x_1 x_2 \dots x_n$}.}

	\item \so{Wenn die Funktion $\phi\left(\mathfrak{x}\right)$
	und die Relation $R\left(\mathfrak{x}, \mathfrak{y}\right)$
	rekursiv sind, dann auch die Relationen $S, T$}

	\begin{equation*}
		\begin{aligned}
			S\left(\mathfrak{x}, \mathfrak{y}\right) \thicksim \left(Ex\right)\left[x \leqq \phi\left(\mathfrak{x}\right) \& R\left(\mathfrak{x}, \mathfrak{y}\right)\right]\\
			T\left(\mathfrak{x}, \mathfrak{y}\right) \thicksim \left[x \leqq \phi\left(\mathfrak{x}\right) \longrightarrow R\left(\mathfrak{x}, \mathfrak{y}\right)\right]
		\end{aligned}
	\end{equation*}
	\so{sowie die Funktion $\psi$}
	\begin{equation*}
		\psi\left(\mathfrak{x}, \mathfrak{y}\right) = \epsilon x\left[x \leqq \phi\left(\mathfrak{x}\right) \& R\left(\mathfrak{x}, \mathfrak{y}\right)\right],
	\end{equation*}
	wobei $\epsilon x F\left(x\right)$ bedeutet: Die kleinste
	Zahl $x$, für welche $F\left(x\right)$ gilt und $0$, falls es
	keine solche Zahl gibt.
\end{enumerate}

Satz I folgt unmittelbar aus der Definition von \glqq rekursiv\grqq. Satz II und III beruhen darauf, daß die
den logischen Begriffen $\overline{\phantom{XX}}$, $\lor$, $=$ entsprechenden zahlentheoretischen Funktionen
\begin{equation*}
	\alpha\left(x\right), \beta\left(x, y\right), \gamma\left(x, y\right)
\end{equation*}
nämlich:\mnote{\customlabel{kommentarAlpha}{\arabic{commentaryNumber}}\noindent\begin{equation*}
	\noindent \thicksim x \equiv \alpha\left(x\right) = \left\{
	\begin{aligned}
		&1, && \text{wenn }\ x = 0 \\
		&0, && \text{sonst}
	\end{aligned} \right.
\end{equation*}}
\begin{equation*}
	\alpha\left(0\right) = 1; \alpha\left(x\right) = 0\text{ für } x \neq 0
\end{equation*}

\mnote{\noindent\begin{equation*}
	\noindent x \wedge b \equiv \beta\left(x, y\right) = \left\{
	\begin{aligned}
		&1, && \text{wenn }\ x \wedge y \\
		&0, && \text{sonst}
	\end{aligned} \right.
\end{equation*}}
\begin{equation*}
	\beta\left(0, x\right) = \beta\left(x, 0\right) = 0; \beta\left(x, y\right) = 1,
	\text{ wenn } x, y \text{ beide } \neq 0 \text{ sind}
\end{equation*}

\mnote{\noindent\begin{equation*}
	\noindent \thicksim(x = y) \equiv \gamma\left(x, y\right) = \left\{
	\begin{aligned}
		&0, && \text{wenn }\ x = y \\
		&1, && \text{sonst}
	\end{aligned} \right.
\end{equation*}}
\begin{equation*}
	\gamma\left(x, y\right) = 0, \text{ wenn } x = y; \gamma\left(x, y\right) = 1, \text{ wenn } x \neq y
\end{equation*}
rekursiv sind, wie man sich leicht überzeugen kann. Der
Beweis für Satz IV ist kurz der folgende: Nach der
Voraussetzung gibt es ein rekursives $\rho\left(\mathfrak{x}, \mathfrak{y}\right)$, so daß:

\begin{equation*}
	R\left(\mathfrak{x}, \mathfrak{y}\right) \thicksim \left[\rho\left(\mathfrak{x}, \mathfrak{y}\right) = 0\right].
\end{equation*}

Wir definieren nun nach dem Rekursionsschema (2) eine
Funktion $\chi\left(\mathfrak{x}, \mathfrak{y}\right)$ folgendermaßen:

\mnote{Zweistufig-rekursive Funktionsdefinition}
\begin{equation*}
	\chi\left(0, \mathfrak{y}\right) = 0
\end{equation*}
\mnote{Gleichbedeutend mit $\chi\left(n, \mathfrak{y}\right) = n \times \chi\left(\left(n - 1\right), \mathfrak{y}\right) \times \alpha\left(a\right)$, für $\alpha$ siehe Kommentar \ref{kommentarAlpha}}
\begin{equation*}
	\chi\left(n + 1, \mathfrak{y}\right) = \left(n + 1\right).a + \chi\left(n, \mathfrak{y}\right).\alpha\left(a\right)\footnote{Wir setzen als bekannt voraus, daß die Funktionen $x + y$ (Addition), $x.y$ (Multiplikation) rekursiv sind.}
\end{equation*}
wobei $a = \alpha\left[\alpha\left(\rho\left(0, \mathfrak{y}\right)\right)\right].\alpha\left[\rho\left(n + 1, \mathfrak{y}\right)\right].\alpha\left[\chi\left(n, \mathfrak{y}\right)\right]$.

$\chi\left(n + 1, \mathfrak{y}\right)$ ist daher entweder
$ = n + 1$ (wenn $a = 1$) oder
$ = \chi\left(n, \mathfrak{y}\right)$ (wenn
$a = 0$)\footnote{Andere Werte als 0 und 1 kann $a$, wie aus der
Definition für $\alpha$ ersichtlich ist, nicht annehmen.}.
Der erste Fall tritt offenbar dann und nur dann ein, wenn
sämtliche Faktoren von $a$ $1$ sind, d.\,h. wenn gilt:

\begin{equation*}
	\overline{R}\left(0, \mathfrak{y}\right) \& R\left(n + 1, \mathfrak{y}\right) \& \left[\chi\left(n, \mathfrak{y}\right) = 0\right]
\end{equation*}

Daraus folgt, daß die Funktion $\chi\left(n, \mathfrak{y}\right)$ (als
Funktion von $n$ betrachtet) 0 bleibt bis zum kleinsten Wert von $n$, für
den $R\left(n, \mathfrak{y}\right)$ gilt, und von
da ab gleich diesem Wert ist (falls schon $R\left(0, \mathfrak{y}\right)$
gilt, ist dementsprechend $\chi\left(n, \mathfrak{y}\right)$ konstant und $ = 0$).
Demnach gilt:

\begin{equation*}
	\psi\left(\mathfrak{x}, \mathfrak{y}\right) = \chi\left(\phi\left(\mathfrak{x}\right), \mathfrak{y}\right)
\end{equation*}
\begin{equation*}
	S\left(\mathfrak{x}, \mathfrak{y}\right) \thicksim R\left[\psi\left(\mathfrak{x}, \mathfrak{y}\right), \mathfrak{y}\right]
\end{equation*}

Die Relation $T$ läßt sich durch Negation auf einen zu
$S$ analogen Fall zurückführen, womit Satz IV bewiesen
ist.

Die Funktionen $x + y, x.y, x^y$, ferner die Relationen
$x < y, x = y$ sind, wie man sich leicht überzeugt,
rekursiv und wir definieren nun, von diesen Begriffen
ausgehend, eine Reihe von Funktionen (Relationen) 1--45,
deren jede au den vorhergehenden mittels der in den Sätzen
I bis IV genannten erfahren definiert ist. Dabei sind
meistens mehrere der nach Satz I bis IV erlaubten
Definitionsschritte in einem zusammengefaßt. Jede der
Funktionen (Relationen) 1--45, unter denen z.\,B. die
Begriffe \glqq\textit{Formel}\grqq, \glqq\textit{Axiom}\grqq, \glqq\textit{unmittelbare Folge}\grqq\ vorkommen,
ist daher rekursiv.\mnote{Es folgt die Liste der Axiome und Definitionen aus diesen Axiomen zum Herleiten der Unvollständigkeitssätze (§§ \refsatz{x/y} -- \refsatz{Bew}).}

\label{genaueformeln}
\begin{enumerate}[1.]
	% 1:
	\item \customlabel{x/y}{1}$x/y \equiv \left(Ez\right) \left[z \leqq x \& x = y.z\right]$%
\mnote{$x/y \equiv \exists z\left(z \leqq x \wedge x = y\times z\right)$, d.\,h. es gibt ein $z \in \mathbb{N}$, für das gilt, dass $y \times z = x$.}%
\footnote{Das Zeichen $\equiv$ wird im Sinne von \glqq Definitionsgleichheit\grqq\ verwendet, vertritt also bei Definitionen entweder $=$ oder $\thicksim$ (im übrigen ist
	die Symbol die \so{Hilbert}sche).}

		$x$ ist teilbar durch $y$\footnote{Überall, wo in den folgenden Definitionen eines der Zeichen $\left(x\right), \left(Ex\right), \epsilon x$ auftritt, ist es von einer Abschätzung für $x$ gefolgt. Diese Abschätzung dient lediglich dazu, um die rekursive Natur des definierten Begriffes (vgl. Satz IV) zu sichern. Dagegen würde sich der Umgang der definierten Begriffe durch Weglassung dieser Abschätzung meistens nicht ändern.}

	% 2:
	\item \customlabel{Prim}{2}$\text{Prim }\left(x\right) \equiv \overline{\left(Ez\right)} \left[z \leqq x \& z \neq 1 \& z \not = x \& x/z\right] \& x > 1$\mnote{$\text{Prim }\left(x\right) \equiv \thicksim \exists z: \left(z \leqq x \wedge z \not= 1 \wedge z \not= x \wedge x/z\right) \wedge x > 1$}

	$x$ ist eine Primzahl.

	% 3:
	\item \customlabel{0 Pr x}{3} $0 Pr\ x \equiv 0$%
\mnote{In der 0 sind keine Primfaktoren}

	$\left(n + 1\right) Pr\ x \equiv \epsilon y \left[y \leqq x \& \text{Prim}\left(y\right) \& x/y \& y > n Pr\ x\right]$%
\mnote{$n Pr x \equiv \exists y: y  \leqq x \wedge\text{Prim } x \wedge x/y \wedge y > (n - 1) Pr\ x$. Beschreibt den $n$ten Primfaktor von $x$, vgl. \refsatz{x/y}, \refsatz{Prim}, \refsatz{0 Pr x}}
	\let\originalfootnote=\thefootnote
	\let\thefootnote=\fnvierunddreissiga

	$n Pr\ x$ ist die $n$-te (der Größe nach) in $x$ enthaltene
		Primzahl\footnote{Für $0 < n \leqq z$, wenn $z$ die Anzahl der verschiedenen
		in $x$ aufgehenden Primzahlen ist. Man beachte, daß für $n = z + 1$ \hspace{0.3cm} $n Pr\ x = 0$ ist!}.
	\let\thefootnote=\originalfootnote
	\setcounter{footnote}{35}

	% 4:
	\item \customlabel{Fakultät}{4}$0! \equiv 1$\mnote{Fakultät, zweistufig-rekursive Funktionsdefinition}

		$\left(n + 1\right)! \equiv \left(n + 1\right).n!$\mnote{Gleich mit $\left(n\right)! \equiv \left(n - 1\right) \times n!$}

	% 5:
	\item \customlabel{Pr}{5}$Pr\ \left(0\right) \equiv 0$

	$Pr\ \left(n + 1\right) \equiv \epsilon y \left[y \leqq \left\{ Pr\ \left(n\right) \right\}! + 1 \& \text{Prim }\left(y\right) \& y > Pr\ \left(n\right)\right] $%
		\mnote{$Pr\ \left(n\right) \equiv \exists y: y \leqq \left(Pr\ \left(n\right)\right)! + 1 \wedge \text{Prim }\left(y\right) \wedge y > Pr\ \left(n\right)$, vgl. §§ \refsatz{Prim}, \refsatz{Fakultät}, \refsatz{Pr}}

	$Pr\ \left(n\right)$ ist die $n$-te Primzahl (der Größe nach).

	% 6:
	\item \customlabel{n Gl x}{6} $n Gl\ x \equiv \epsilon y \left[y \leqq x \& x/\left(n Pr x\right)^y \& \overline{x/n Pr x)^{y + 1}}\right]$%
		\mnote{$n Gl\ x \equiv \exists y: y \leqq x \wedge x/\left(n Pr x\right)^y \wedge \thicksim \left(\left(x / n Pr x\right)^{x + 1}\right)$, vgl. §§ \refsatz{x/y}, \refsatz{0 Pr x}}

	$n Gl\ x$ ist das $n$-te Glied der der Zahl $x$
	zugeordneten Zahlenreihe (für $n > 0$ und $n$ nicht
	größer als die Länge dieser Reihe).

	% 7:
	\item \customlabel{l(x)}{7} $l\left(x\right) \equiv \epsilon y \left[y \leqq x \& y Pr x > 0 \& \left(y + 1\right) Pr x = 0\right]$%
\mnote{$l\left(x\right) \equiv \exists y: y \leqq x \wedge y Pr x > 0 \wedge \left(y + 1\right) Pr x = 0$}

	$l\left(x\right)$ ist die Länge der $x$ zugeordneten Zahlenreihe.

	% 8:
	\item \customlabel{Konkat}{8}$x \pmstar y \equiv
		\left\{
			z \leqq \left[ Pr \left(l\left(x\right) + l\left(y\right)\right)\right]^{x + 1} \& \right.$
	$		\left(n\right) \left[n \leqq l\left(x\right) \longrightarrow n Gl\ z = n Gl\ x\right] \& $
	$\left.		\left(n\right) \left[0 < n \leqq l\left(y\right) \longrightarrow \left(n + l\left(x\right)\right) Gl z = n Gl y\right]
		\right\}$

	$x \pmstar y$ entspricht der Operation des \glqq Aneinanderfügens\grqq\ zweier endlicher Zahlenreihen.\mnote{Konkatinieren von Zahlen}

	% 9:
	\item  $R\left(x\right) \equiv 2^x$\mnote{Für alle $n \in \mathbb{N}$ erzeugt
		das eine eineindeutige Zahl, die immer wieder \frq zurückkonvertiert\flq\ werden
		kann zu ihrer Ursprungszahl. Diese Methode ist ein Spezialfall der generalisierten
		Methode $p(n)^x$, wobei $p(n)$ die Position im Zeichenstring kodiert ($n$) und
		$p(x)$ die $x$te Primzahl zurückgibt. Multiplikativ zusammengefügt ist eine
		eineindeutige Zerlegung von Strings in eine Zahlenkette möglich.}

	$R\left(x\right)$ entspricht der nur aus der Zahl $x$ bestehenden Zahlenreihe (für $x > 0$).

	% 10:
	\item \customlabel{E(x)}{10} $E\left(x\right) \equiv R\left(11\right) \pmstar x \pmstar R\left(13\right)$\mnote{$E\left(x\right) \equiv \text{\glqq}(\text{\grqq. } x \text{. \glqq})\text{\grqq}$, vgl. § \refsatz{Konkat}}

	$E\left(x\right)$ entspricht der Operation des \glqq Einklammerns\grqq\ [11 und 13 sind den Grundzeichen
	\glqq(\grqq\ und \glqq)\grqq\ zugeordnet].

	%11:
	\item \customlabel{n Var x}{11}$n\text{ Var } x \equiv \left(Ez\right)\left[13 < z \leqq x \& \text{Prim }\left(z\right) \& x = z^n\right] \& n \neq 0$%
\mnote{$n\text{ Var } x \equiv \exists z: \left[\text{\glqq)\grqq} < z \leqq x \wedge \text{Prim }\left(z\right) \wedge x = z^n\right] n \not= 0$, die 13 ist die Gödelnummer für \glqq)\grqq, vgl. § \refsatz{Prim}}

	%12:
	$x$ ist eine \textit{Variable $n$-ten Typs}.

	\item $\text{Var }\left(x\right) \equiv \left(En\right)\left[n \leqq x \& n \text{ Var } x\right]$%
\mnote{$\text{Var }\left(x\right) \equiv \exists n: \left(n \leqq x \wedge \left(n \text{ Var } x\right)\right)$, vgl. § \refsatz{n Var x}}

	$x$ ist eine \textit{Variable}.

	% 13:
	\item \customlabel{Neg}{13} $\text{Neg }\left(x\right) \equiv R\left(5\right) \pmstar E\left(x\right)$\mnote{$\text{Neg }\left(x\right) \equiv \thicksim x$, vgl. \refsatz{Konkat}.}

	$\text{Neg }\left(x\right)$ ist die \textit{Negation} von
	$x$.

	% 14:
	\item \customlabel{Dis}{14}$x \text{ Dis } y \equiv E\left(x\right) \pmstar R\left(7\right) \pmstar E\left(y\right)$%
\mnote{$x \text{ Dis } y \equiv x\text{. \glqq}\lor\text{\grqq. } y$, vgl. § \refsatz{Konkat}}

	$x \text{ Dis } y$ ist die \textit{Disjunktion} aus $x$ und $y$.

	% 15:
	\item \customlabel{Gen}{15} $x \text{ Gen } y \equiv R\left(x\right) \pmstar R\left(9\right) \pmstar E\left(y\right)$%
		\mnote{$x \text{ Gen } y \equiv x \text{. \glqq}\Pi\text{\grqq. } y$, oder in modern:
		$x \text{ Gen } y \equiv \forall x: y$, vgl. § \refsatz{Konkat}}

	$x \text{ Gen } y$ ist die \textit{Generalisation} von $y$ mittels der Variablen $x$ (vorausgesetzt, daß $x$ eine \textit{Variable} ist).

	%16:
	\item \customlabel{0 N x}{16}$0 N x \equiv x$\mnote{Zweistufig-rekursive Funktion zum Vorsetzen des Zeichens \glqq$f$\grqq}

	$\left(n + 1\right) N x \equiv R\left(3\right) n N x$%
\mnote{$\left(n\right) N x \equiv \text{\glqq}f\text{\grqq.} \left(n - 1\right) N x$, die 3 ist das \glqq $f$\grqq, vgl. § \refsatz{Konkat}}

	$n N x$ entspricht der Operation \glqq$n$-maliges
	Vorsetzen des Zeichens \glq$f$\grq\ vor $x$\grqq.

	%17:
	\item \customlabel{Z(n)}{17}$Z\left(n\right) \equiv n N\left[R\left(1\right)\right]$%
\mnote{Konvertiert \glqq normale\grqq\ Zahlen zu einem Zahlzeichen. So wird die \glqq 5 \grqq\ zu $ffff1$.}
	$Z\left(n\right)$ ist das \textit{Zahlzeichen} für die Zahl $n$.

	\let\originalfootnote=\thefootnote
	\let\thefootnote=\fnvierunddreissigb

	% 18:
	\item \customlabel{Typ1'}{18}$$
		\text{Typ}_1'\left(x\right) \equiv \left(Em, n\right) \left\{m, n\leqq x \& \left[m = 1 \lor 1\textit{ Var } m\right] \right.
	$$\\[\spacebetweenbreakedequations]
	$$\left.\& x = n N \left[R\left(m\right)\right]\right\}
\footnote{$m, n \leqq x$ steht für: $m \leqq x \& n \leqq x$ (ebenso für mehr als 2 Variable).}$$\\[\spaceafterbreakedequation]%
\mnote{$\text{Typ}_1\left(x\right) \equiv \exists m \exists n: n \leqq x \wedge m \leqq x \wedge \left(m = 1 \lor 1 Var\ m\right) \wedge x = n N\left[R\left(m\right)\right]$}
	\let\thefootnote=\originalfootnote
	\setcounter{footnote}{34}

	$x$ ist \textit{Zeichen ersten Typs}.

	% 19:
	\item \customlabel{Typ}{19}$$\text{Typ}_n\left(x\right) \equiv \left[n = 1 \& \text{Typ}_1'\left(x\right)\right] \lor \left[n > 1 \& \right.
	$$\\[\spacebetweenbreakedequations]
	$$\left.\left(Ev\right) \left\{v \leqq x \& n\text{ Var } v \& x = R\left(v\right)\right\}\right]$$\\[\spaceafterbreakedequation]

	$x$ ist \textit{Zeichen $n$-ten Typs}.\mnote{Siehe § \refsatz{Typ1'} für Spezialfall $n = 1$}

	%20:
	\item $$
		Elf\left(x\right) \equiv \left(E y, z, n\right) \left[y, z, n \leqq x \& \text{Typ}_n\left(y\right)\right.
	$$\mnote{$Elf\left(x\right) \equiv \exists y: \exists z: \exists n: y, z, n \leqq x \wedge \text{Typ}_n\left(y\right) \wedge \text{Typ}_{n + ¹}\left(z\right) \wedge x = z \pmstar E\left(y\right)$, vgl. §§ \refsatz{Konkat}, \refsatz{Typ}, \refsatz{E(x)}.}\\[\spacebetweenbreakedequations]
	$$\left.\& \text{Typ}_{n + 1}\left(z\right) \& x = z \pmstar E\left(y\right)\right]$$\\[\spaceafterbreakedequation]

	$x$ ist \textit{Elementarformel}\mnote{Vgl. Kommentar \ref{elementarformeln}}.

	%21:
	\item $Op\left(x\ y\ z\right) \equiv x = \text{Neg}\left(y\right) \lor x = y \text{ Dis } z \lor\allowbreak \left(Ev\right) \left[v \leqq x \& \text{Var}\left(x\right) \& x = v \text{ Gen } y\right]$

	% 22:
	\item $$
		FR\left(x\right) \equiv \left(n\right) \left\{ 0 < n \leqq l\left(x\right) \longrightarrow\allowbreak Elf\left(n Gl\ x\right) \lor\right.
	$$\\[\spacebetweenbreakedequations]
	$$
		\left.\left(Ep, q\right)\allowbreak \left[0 < p, q < n \&\allowbreak Op\left(n Gl\ x, p Gl\ x, q Gl\ x\right)\right]\right\} \&
	$$\\[\spacebetweenbreakedequations]
	$$
		l\left(x\right) > 0
	$$\\[\spaceafterbreakedequation]

	$x$ ist eine Reihe von \textit{Formeln}, deren jede entweder
	\textit{Elementarformel} ist oder aus vorhergehenden durch die Operationen der \textit{Negation},
	\textit{Disjunktion}, \textit{Generalisation} hervorgeht.

	% 23:
	\item \customlabel{Form}{23}$$\text{Form}\left(x\right) \equiv \left(En\right) \left\{n \leqq \left(Pr \left[l\left(x\right)^2\right]\right)^{x.\left[l\left(x\right)\right]^2}\right.
		$$\\[\spacebetweenbreakedequations]
		$$\left.\& FR\left(n\right) \& x = \left[l\left(n\right)\right] Gl n\right\}\footnotemark$$\footnotetext{Die Abschätzung $n \leqq \left(Pr \left[l\left(x\right)^2\right]\right)^{x l\left(x\right)^2}$
		erkennt man so: Die Länge der kürzesten zu $x$ gehörigen Formelreihe
		kann höchstens gleich der Anzahl der Teilformeln von $x$ sein.
		Es gibt aber höchstens $l\left(x\right)$ Teilformeln der Länge 1,
		höchstens $l\left(x\right) - 1$ der Länge 2 usw., im Ganzen als
		höchstens $\frac{l\left(x\right)\left[l\left(x\right) + 1\right]}{2} \leqq l\left(x\right)^2$.
		Die Primzahlen aus $n$ können also sämtlich kleiner als
		$Pr\left\{\left[l\left(x\right)\right]^x\right\}$ angenommen werden,
		ihre Anzahl $\leqq l\left(x\right)^2$ und ihre Exponenten (welche Teilformeln von $x$ sind) $\leqq x$.}

	$x$ ist \textit{Formel} (d.\,h. letztes Glied einer \textit{Formelreihe $n$}).

	% 24:
	\item \customlabel{Geb}{24}$$
		v \text{ Geb } n, x \equiv \text{Var}\left(v\right) \& \text{Form}\left(x\right) \&
	$$\\[\spacebetweenbreakedequations]
	$$
	\left(Ea, b, c\right) \left[a, b, c \leqq x \& x a = \pmstar \left(v \text{ Gen } b\right) \pmstar c\right.
	$$\\[\spacebetweenbreakedequations]
	$$\left.\& \text{Form}\left(b\right) \& l\left(a\right) + 1 \leqq n \leqq l\left(a\right) + l\left(v \text{ Gen } b\right)\right]
	$$\\[\spaceafterbreakedequation]

	Die \textit{Variable} $v$ ist in $x$ an $n$-ter Stelle \textit{gebunden}.

	% 25:
	\item \customlabel{v Fr n, x}{25}$v Fr\ n, x \equiv \text{Var}\left(v\right) \& \text{Form}\left(x\right) \& v = n Gl\ x \& n \leqq l\left(x\right) \& v \text{ Geb } n, x$

	Die \textit{Variable} $v$ ist in $x$ an $n$-ter Stelle \textit{frei}.

	% 26:
	\item $v Fr\ x \equiv \left(En\right) \left[n \leqq l\left(x\right) \& v Fr\ n, x\right]$%
\mnote{$v Fr\ x \equiv \exists n: n \leqq l\left(x\right) \wedge v Fr\ n, x$, vgl. §§ \refsatz{l(x)},
\refsatz{v Fr n, x}. Macht aus dem Satz \refsatz{v Fr n, x}, der herausfindet, ob $v$ in $x$ an $n$ter
Stelle frei ist die generalisierte Aussage: die Variable $v$ befindet sich überhaupt in $x$ oder nicht.}

	$v$ kommt in $x$ als \textit{freie Variable} vor.

	% 27:
	\item $$
		Su\ x \Spvek{n; y} \equiv \epsilon z \left\{ z \leqq \left[Pr\left(l\left(x\right) + l\left(y\right)\right)\right]^{x + y} \& \left[\left(Eu, v\right) u, v \leqq x \& \right.\right.
	$$\\[\spacebetweenbreakedequations]
	$$
		\left.\left.x = u \pmstar R\left(n Gl\ x\right) v \& z = u \pmstar y \pmstar v \& n = l\left(u\right) + 1\right]\right\}
	$$\\[\spacebetweenbreakedequations]

	$Su\ x \Spvek{n; y}$ entsteht aus $x$, wenn man an Stelle des $n$-ten Gliedes von $x$ $y$ einsetzt (vorausgesetzt, daß $0 < n \leqq l\left(x\right)$).

	%28:
	\item \customlabel{n St v, x}{29}$$
		0 St\ v, x \equiv \left\{ n \leqq l\left(x\right) \& v Fr\ n, x \right.
	$$\\[\spacebetweenbreakedequations]
	$$\left.\& \overline{\left(Ep\right)} \left[n < p \leqq l\left(x\right) \& v Fr\ p x\right]\right\}
	$$\\[\spaceafterbreakedequation]

	$$
	\left(k + 1\right) St\ v, x \equiv \epsilon n \left\{n < k St\ v, x \& v Fr\ n, x \right.
	$$\\[\spacebetweenbreakedequations]
	$$
		\left.\& \overline{\left(Ep\right)} \left[n < p < k St\ v, x \& v Fr\ p, x\right]\right\}
	$$\\[\spaceafterbreakedequation]

	$k St\ v, x$ ist die $k + 1$-te Stelle in $x$ (vom Ende der \textit{Formel} $x$ an gezählt), an der $v$ in $x$ \textit{frei} ist (und 0, falls es keine solche Stelle gibt).

	%29:
	\item $A\left(v, x\right) \equiv \epsilon n\left\{ n \leqq l\left(x\right) \& n St\ v, x = 0\right\}$%
\mnote{$A\left(v, x\right) \equiv \exists n: n \leqq l\left(x\right) \wedge n St\ v, x = 0$, vgl. \refsatz{n St v, x}, \refsatz{l(x)}}

	$A\left(v, x\right)$ ist die Anzahl der Stellen, an denen $v$ in $x$ \textit{frei} ist.

	%30:
	\item $Sb_0 \left(x\substack{v\\y }\right) \equiv \alpha$%
\mnote{Rekursive Funktion zum Ersetzen von Variablen, letzter Rekursionsschritt ist immer die 0 (daher spezial definiert)}

	$Sb_{k + 1} \left(x\substack{v\\y}\right) \equiv Su\ \left[Sb_k\left(x\substack{v\\y}\right)\right] \left(\substack{k\ St\ v, x\\y}\right)$%
\mnote{Expandiert die Variablen bis in die unterste Formelebene zu ihrem Ersatz}

	% 31
	\item $Sb\left(x\substack{v\\y}\right) \equiv Sb_{A\left(v, x\right)}\left(x\substack{v\\y}\right)$\footnote{Falls $v$ keine \textit{Variable} oder $x$ keine \textit{Formel} ist, ist $Sb\left(x\substack{v\\y}\right) = x$.}

	$Sb\left(x\substack{v\\y}\right)$ ist der oben definierte Begriff $Subst\ a\left(\substack{v\\b}\right)$\footnote{Statt $Sb\left[Sb\left(x\substack{v\\y}\right)\substack{w\\z}\right]$ schreiben wir: $Sb\left(x\substack{v\ u\\y\ z}\right)$ (analog für mehr als zwei Variable).}.

%32:
	\customlabel{Logische Operatoren}{32}
	\item \customlabel{Imp}{32a}$x \text{ Imp } y \equiv \left[\text{Neg }\left(x\right)\right] \text{ Dis } y $\mnote{$x \text{ Imp } y \equiv \left(\thicksim x\right) \lor y$. Bedeutet: $x \supset y$}

	\customlabel{Con}{32b}$x \text{ Con } y \equiv \text{ Neg } \left\{\left[\text{Neg }\left(x\right)\right]\text{ Dis }\left[\text{Neg }\left(y\right)\right]\right\}$\mnote{$x \text{ Con } y \equiv \left(x \supset y\right) \lor \thicksim y$. Bedeutet $x \lor y$}

	\customlabel{Aeq}{32c}$x \text{ Aeq } y \equiv \left(x \text{ Imp } y\right) \text{ Con } \left(y \text { Imp } x\right)$%
	\mnote{$x \text{ Aeq } y \equiv \left(x \supset y\right) \wedge \left(y \supset x\right)$. Bedeutet: $x \text{ iff } y$}

	\customlabel{Ex}{32d}$v \text { Ex } y \equiv \text{Neg } \left\{v \text{ Gen } \left[\text{Neg } \left(y\right)\right]\right\}$%
\mnote{$v \text{ Ex } y \equiv \thicksim \left(\forall v: \lnot y\right)$. Bedeutet \glqq alle $y$ in $v$ sind wahr\grqq, vgl. §§ \refsatz{Neg}, \refsatz{Gen}}

%33:
	\item \customlabel{Th}{33} $$
		n Th\ x \equiv \epsilon y \left\{ y \leqq x^{\left(x^n\right)} \& \left(k\right) \left[k \leqq l\left(x\right) \longrightarrow \right.\right.
	$$\\[\spacebetweenbreakedequations]
	$$
		\left.\left.\left(k Gl\ x \leqq 13 \& k Gl\ y = k Gl\ x\right) \lor\right.\right.
	$$\\[\spacebetweenbreakedequations]
	$$
		\left.\left.\left(k Gl\ x > 13 \& k Gl\ y = k Gl\ x.\left[1 Pr\ \left(k Gl\  x\right)\right]^n\right)\right]\right\}
	$$\\[\spaceafterbreakedequation]

	$n Th\ x$ ist die \textit{$n$-te Typenerhöhung} von $x$ (falls $x$ und $n Th\ x$ \textit{Formeln} sind).

	Den Axiomen I, 1 bis 3 entsprechen drei bestimmte Zahlen, die wir mit $z_1, z_2, z_3$ bezeichnen un wir definieren:

	%34:
	\item \customlabel{Z - Ax(x)}{34}$Z - Ax\left(x\right) \equiv \left(x = z_1 \lor x = z_2 \lor x = z_3\right)$

	%35:
	\item $A_1 - Ax\left(x\right) \equiv \left(Ey\right) \left[y \leqq x\& \text{Form }\left(y\right) \& x = \left(y \text{ Dis } y\right) \text{ Imp } y\right]$%
\mnote{$A_1 - Ax\left(x\right) \equiv \exists y: y \leqq x \wedge \text{Form }\left(y\right) \wedge x = \left(y \lor y\right) \supset y$, vgl. § \refsatz{Dis}, \refsatz{Form}, \refsatz{Form}, \refsatz{Dis}, \refsatz{Imp}, \refsatz{Z - Ax(x)}}
% Was bringt y \lor y???

	$x$ ist eine durch Einsetzung in das Axiomenschema II, 1 entstehende \textit{Formel}. Analog werden $A_2 - Ax, A_3 - Ax, A_4 - Ax$ entsprechend den Axiomen II, 2 bis 4 definiert.

	%36:
	\item $A - Ax \left(x\right) \equiv A_1 - Ax\left(x\right) \lor A_2 - Ax\left(x\right) \lor A_3 - Ax\left(x\right) \lor A_4 - Ax\left(x\right)$

	$x$ ist eine durch Einsetzung in ein Aussagenaxiom entstehende \textit{Formel}.

	%37:
	\item $$Q\left(z, y, v\right) \equiv \overline{\left(En, m, w\right)} \left[n \leqq l\left(y\right) \& m \leqq l\left(z\right) \& w \leqq z\right. \&
	$$\\[\spacebetweenbreakedequations]
	$$ \left.w = m Gl z \& w \text{ Geb } n, y \& v Fr\ n, y\right]$$\\[\spaceafterbreakedequation]

	$z$ enthält keine Variable, die in $y$ an einer Stelle
	\textit{gebunden} ist, an der $v$ \textit{frei} ist.

	%38:
	\item $$
		L_1 - Ax\left(x\right) \equiv \left(Ev, y, z, n\right) \left\{v, y, z, n \leqq x \& n \text{ Var } v \&\right.
	$$\\[\spacebetweenbreakedequations]
	$$
		\left.\text{Typ}_n\left(z\right) \& \text{Form}\left(y\right) \& Q\left(z, y, v\right) \&\right.
	$$\\[\spacebetweenbreakedequations]
	$$
		\left.x = \left(v \text{ Gen } y\right) \text{ Imp } \left[Sb \left(y \substack{v\\z}\right)\right]\right\}
	$$\\[\spaceafterbreakedequation]

	$x$ ist eine aus dem Axiomenschema III, 1 durch Einsetzung entstehende \textit{Formel}.

	%39:
	\item $$
		L_2 - Ax\left(x\right) \equiv \left(Ev, q, p\right) \left\{v, q, p \leqq x \& \text{Var}\left(v\right)\right.
	$$\\[\spacebetweenbreakedequations]
	$$
		\left.\& \text{Form}\left(p\right) \& v Fr\ p \& \text{Form}\left(x\right) \&\right.
	$$\\[\spacebetweenbreakedequations]
	$$
		\left.x = \left[v \text{ Gen } \left(p\text{ Dis } q\right)\right]\right\}
	$$\\[\spaceafterbreakedequation]
	$x$ ist eine aus dem Axiomenschema III, 2 durch Einsetzung entstehende \textit{Formel}.

	%40:
	\item $$
		R - Ax\left(x\right) \equiv \left(Eu, v, y, n\right) \left[u, v, y, n \leqq x \& n \text{ Var } v \& \right.
	$$\\[\spacebetweenbreakedequations]
	$$
		\left(n + 1\right) \text{ Var } u \&  u Fr\ y \& \text{Form }\left(y\right) \&
	$$\\[\spacebetweenbreakedequations]
	$$
		\left.x = u\text{ Ex } \left\{v \text{Gen } \left[\left[R\left(u\right) \pmstar E\left(R\left(v\right)\right)\right] \text{ Aeq } y \right]\right\} \right]
	$$\\[\spaceafterbreakedequation]

	$x$ ist eine aus aus dem Axiomenschema IV, 1 durch
	Einsetzung entstehende \textit{Formel}.

	Dem Axiom V, 1 entspricht eine bestimmte Zahl $z_4$ und wir definieren:

	%41:
	\item $M - Ax\left(x\right) \equiv \left(En\right) \left[n \leqq x \& x = n Th\ z_4\right]$%
\mnote{$M - Ax\left(x\right) \equiv \exists n: n \leqq x \wedge x = n Th\ z_4$, vgl. § \refsatz{Th}}

	%42:
	\item $Ax\left(x\right) \equiv Z - Ax\left(x\right) \lor A - Ax\left(x\right) \lor L_1 - Ax\left(x\right) \lor L_2 - Ax\left(x\right) \lor R - Ax\left(x\right) \lor M - Ax\left(x\right)$

	$x$ ist ein \textit{Axiom}.

	\item $Fl\left(x\ y\ z\right) \equiv y = z \text{ Imp } x \lor \left(Ev\right) \left[ v \leqq x \& \text{ Var }\left(v\right) \& x = v \text{ Gen } y\right]$

	$x$ ist \textit{unmittelbare Folge} aus $y$ und $z$.

	%44:
	\item \customlabel{Bw}{44}$$
		Bw\left(x\right) \equiv \left(n\right)\left\{0 < n \leqq l\left(x\right) \longrightarrow Ax\left(n Gl\ x\right) \lor \right.
	$$\\[\spacebetweenbreakedequations]
	$$	\left.\left(Ep, q\right) \left[0, < p, q < n \& Fl\left(n Gl\ x, p Gl x, q Gl x\right)\right]\right\}
	$$\\[\spacebetweenbreakedequations]
	$$
		\& l\left(x\right) > 0
	$$\\[\spaceafterbreakedequation]

	$x$ ist eine \textit{Beweisfigur} (eine endliche Folge von \textit{Formeln}, deren jede entweder \textit{Axiom} oder \textit{unmittelbare Folge} aus der vorhergehenden ist).

	% nGl x ist das n-te Glied der der Zahl x zugeordneten Zahlenreihe
	%(für n > 0 und n nicht größer als die Länge dieser Reihe).
	% 45:
	\item \customlabel{x B y}{45}$x B\ y \equiv Bw\left(x\right) \& \left[ l\left(x\right) \right] Gl x = y$%
		\mnote{$x$ ist Beweis für $y \equiv \left(x \text{ ist Beweisfigur } \wedge\allowbreak \left(\text{ die Länge von } x \text{ das n}-\text{te Glied der Zahl } x\right) = y\right)$, vgl. §§ \refsatz{l(x)}, \refsatz{n Gl x}, \refsatz{Bw}}

	$x$ ist ein \textit{Beweis} für die \textit{Formel} $y$.

	%46:
	\item \customlabel{Bew}{46}$\text{Bew }\left(x\right) \equiv \left(Ey\right) y B x$\mnote{$\text{Bew } \equiv \exists y: y B x$, es existiert ein Beweis $y$ für $x$, vgl. § \refsatz{x B y}}

	$x$ ist eine \textit{beweisbare Formel}. [$\text{Bew }\left(x\right)$ ist der einzige unter den Begriffen 1--46, von dem nicht behauptet werden kann, er sei rekursiv.]
\end{enumerate}

Die Tatsache, die man vage so formulieren kann: Jede rekursive Relation
ist innerhalb des Systems $P$ (dieses inhaltlich gedeutet) definierbar, wird,
ohne auf eine inhaltliche Deutung der Formeln\mnote{Die inhaltliche Deutung ist keine,
die aus dem Formalismus entspringen kann. Erst in der inhaltlichen Deutung z.\,B.
wird die Wahrheit der Aussage des Satzes $n$, dass $n$ nicht beweisbar ist,
als wahr erkannt; aus dem Formalismus allein heraus geht das nicht.}
aus $P$ Bezug zu nehmen,
durch folgenden Satz exakt ausgedrückt:

Satz V: Zu jeder rekursiven Relation $R\left(x_1 \dots x_n\right)$
gibt es ein $n$-stelliges \textit{Relationszeichen} $r$
(mit den \textit{freien Variablen}\footnote{Die \textit{Variablen}
$u_1 \dots u_n$ können willkürlich vorgegeben werden. Es gibt
z.\,B. immer ein $r$ mit den \textit{freien Variablen} 17, 19, 23 \dots usw.,
für welches (3) und (4) gilt.}
$u_1, u_2 \dots u_n$), so, dass für alle Zahlen-$n$-tupel
$x_1 \dots x_n$ gilt:

% S 15
\begin{equation}
	R\left(x_1 \dots x_n\right) \longrightarrow \text{Bew}\left[Sb\left(r\substack{u_1 \dots u_n\\ Z\left(x_1\right) \dots Z\left(x_n\right)}\right)\right]
\end{equation}

\begin{equation}
	\overline{R}\left(x_1 \dots x_n\right) \longrightarrow \text{Bew }\left[\text{Neg } Sb\left(r\substack{u_1 \dots u_n \\ Z\left(x_1\right) \dots Z\left(x_n\right)}\right)\right]
\end{equation}

Wir begnügen uns hier damit, den Beweis dieses Satzes, da er keine prinzipiellen Schwierigkeiten bietet und ziemlich umständlich ist, in Umrissen anzudeuten\footnote{Satz V beruht natürlich darauf, daß bei einer rekursiven Relation $R$ für jedes $n$-tupel von Zahlen \so{aus den Axiomen des Systems $P$} entscheidbar ist, ob dieses Relation $R$ besteht oder nicht.}.
Wir beweisen den Satz für alle Relationen $R\left(x_1 \dots x_n\right)$ der Form: $x_1 = \phi\left(x2 \dots x_n\right)$\footnote{Daraus folgt sofort seine Geltung für jede rekursive Relation, da eine solche Gleichbedeutung ist mit $0 = \phi\left(x_1 \dots x_n\right)$, wo $\phi$ rekursiv ist.}
(wo $\phi$ eine rekursive Funktion ist) und wenden vollständige Induktion nach der Stufe von $\phi$ an. Für die Funktionen erste Stufe (d.\,h. Konstante und die Funktion $x + 1$) ist der Satz trivial. Habe also
$\phi$ die $m$-te Stufe. Es entsteht aus Funktionen niedrigerer Stufe $\phi_1 \dots \phi_k$ durch die Operationen der Einsetzung oder der rekursiven Definition. Da für $\phi_1 \dots \phi_k$ nach induktiver Annahme bereits alles bewiesen ist, gibt es zugehörige \textit{Relationszeichen} $r_1 \dots r_k$, so daß (3), (4) gilt. Die Definitionsprozesse, durch die $\phi$ aus $\phi_1 \dots \phi_k$ entsteht (Einsetzung und rekursive Definition), können sämtlich im System $P$ formal nachgebildet werden. Tut man dies, so erhält man aus $r_1 \dots r_k$ ein neues \textit{Relationszeichen} $r$\footnote{Bei der genauen Durchführung dieses Beweises wird natürlich $r$ nicht auf dem Umweg über die inhaltliche Deutung, sondern durch seine rein formale Beschaffenheit definiert.},
für welches man die Geltung von (3), (4) unter der Verwendung der induktiven Annahmen ohne Schwierigkeit beweisen kann. Ein \textit{Relationszeichen} $r$, welches auf diesem Wege einer rekursiven Relation zugeordnet ist\footnote{Welches also, inhaltlich gedeutet, das Bestehen dieser Relation ausdrückt.},
soll rekursiv heißen.

Wir kommen nun ans Ziel unserer Ausführungen. Sei $\chi$ eine beliebige Klasse von
\textit{Formeln}. Wir bezeichnen mit $\text{Flg}\left(x\right)$ (Folgerungsmenge von
$\chi$) die kleinste Menge von \textit{Formeln}, die alle \textit{Formeln} aus $\chi$
und alle \textit{Axiome} enthält und gegen die Relation
\glqq\textit{unmittelbare Folge}\grqq\ abgeschlossen ist. $\chi$ heißt
$\omega$-widerspruchsfrei, wenn es kein \textit{Klassenzeichen} $a$ gibt, so daß:
\begin{equation*}
	\left(n\right)\left[Sb\left(a\substack{v\\ Z\left(n\right)}\right) \epsilon \text{ Flg } \left(x\right)\right] \& \left[\text{Neg }\left(v\text{ Gen } a\right)\right] \epsilon \text{ Flg }\left(\chi\right)
\end{equation*}
wobei $v$ die \textit{freie Variable} des \textit{Klassenzeichens} $a$ ist.

Jedes $\omega$-widerspruchsfreie System ist selbstverständlich auch widerspruchsfrei.\mnote{Widerspruchsfreiheit von
nicht $\omega$-widerspruchsfreien Systemen heißt, dass allein die Axiome widerspruchsfrei sind.} Es gilt aber,
wie später gezeigt werden wird, nicht das Umgekehrte.

Das allgemeine Resultat über die Existenz unentscheidbarer Sätze lautet:

Satz VI: \so{Zu jeder $\omega$-widerspruchsfreien rekursiven Klasse $\chi$ von} \textit{Formeln} \so{gibt es
rekursive} \textit{Klassenzeichen} $r$, \so{so daß weder $v \text{ Gen } r$
noch $\text{Neg }\left(v \text{ Gen } r\right)$ zu
$\text{Flg }\left(\chi\right)$ gehört (wobei $v$ die} \textit{freie Variable} \so{aus $r$ ist)}.

Beweis: Sei $\chi$ eine beliebige rekursive $\omega$-widerspruchsfreie Klasse von \textit{Formeln}. Wir definieren:

\begin{equation}
	Bw_x\left(x\right) \equiv \left(n\right) \left[ n \leqq l\left(x\right) \longrightarrow Ax\left(n Gl\ x\right) \lor \left(n Gl\ x\right) \epsilon \chi \lor\right.
\end{equation}
\begin{equation*}
	\left.\left(Ep, q\right) \left\{0 < p, q < n \& Fl\left(n Gl\ x, p Gl x, q Gl x\right)\right\}\right] \& l\left(x\right) > 0
\end{equation*}
(vgl. den analogen Begriff 44)

\begin{equation}
	x B_x y \equiv Bw_x\left(x\right) \& \left[l\left(x\right)\right] Gl x = y
\end{equation}
\begin{equation}
	\tag{6.1}
	\text{Bew}_x\left(x\right) \equiv \left(Ey\right) y B_x x
\end{equation}
(vgl. die analogen Begriffe 45, 46).

Es gilt offenbar:

\begin{equation}
	\left(x\right) \left[\text{Bew}_x\left(x\right) \thicksim x \epsilon \text{Flg}\left(\chi\right)\right]
\end{equation}
\begin{equation}
	\left(x\right) \left[\text{Bew}\left(x\right) \longrightarrow \text{Bew}_x\left(x\right)\right]
\end{equation}

Nun definieren wir die Relation:

\begin{equation}
	\tag{8.1}
	Q\left(x, y\right) \equiv \overline{x B\chi\left[ Sb\left(y\substack{19\\ Z\left(y\right)}\right)\right]}.
\end{equation}

Da $x B_x y$ [nach (6), (5)] und $Sb\left(y\substack{19\\ Z\left(y\right)}\right)$ (nach Def. 17, 31) rekursiv sind, so auch $Q\left(xy\right)$. Nach Satz V und (8) gibt es also ein \textit{Relationszeichen} $q$ (mit den \textit{freien Variablen} 17, 19),\mnote{17 entspricht $x_1$, 19 entspricht $y_1$.} so daß gilt:

\begin{equation}
	\overline{x B_\chi\left[Sb\left(y\substack{19\\ Z\left(y\right)}\right)\right]} \longrightarrow \text{Bew}_{\chi} \left[Sb \left(q \substack{17\\ Z\left(x\right)}\substack{19\\ Z\left(y\right)}\right)\right]
\end{equation}

\begin{equation}
	x B_\chi \left[Sb\left(y\substack{19\\ Z\left(y\right)}\right)\right] \longrightarrow \text{Bew}_\chi \left[\text{Neg }Sb\left(q\substack{17\\ Z\left(x\right)}\substack{19\\ Z\left(y\right)}\right)\right]
\end{equation}

Wir setzen:
\mnote{17 ist die erste \textit{freie Variable} ($x_1$)}
\begin{equation}
	p = 17 \text{ Gen } q
\end{equation}
($p$ ist ein \textit{Klassenzeichen} mit der \textit{freien Variablen} 19) und

\begin{equation}
	r = Sb\left(q\substack{19\\ Z\left(p\right)}\right)
\end{equation}
($r$ ist ein rekursives \textit{Klassenzeichen} mit der
 \textit{freien Variablen} 17\footnote{$r$ entsteht ja
aus dem rekursiven \textit{Relationszeichen} $q$ durch
Ersetzen einer \textit{Variablen} durch eine Bestimmte Zahl $\left(p\right)$.}).
Dann gilt:
\begin{equation}
	Sb\left(p\substack{19\\ Z\left(p\right)}\right) = Sb\left(\left[17\text{ Gen } q\right]\substack{19\\ Z\left(p\right)}\right) = 17\text{ Gen } Sb\left(q\substack{19\\ Z\left(p\right)}\right) = 17\text{ Gen } r\footnote{Die Operationen $\text{Gen}$, $Sb$ sind natürlich immer vertauschbar, falls sie sich auf verschiedene \textit{Variablen} beziehen.}
\end{equation}
[wegen (11) und (12)] ferner:

\begin{equation}
	Sb\left(q\substack{17\\ Z\left(x\right)}\substack{19\\ Z\left(p\right)}\right) = Sb\left(r\substack{17\\ Z\left(x\right)}\right)
\end{equation}
[nach (12)]. Setzt man nun in (9) und (10) $p$ für $y$ ein, so entsteht unter Berücksichtigung von (13) und (14):

\begin{equation}
	\overline{x B_\chi\left(17\text{ Gen } r\right)} \longrightarrow \text{Bew}_\chi \left[Sb\left(r\substack{17\\ Z\left(x\right)}\right)\right]
\end{equation}

\begin{equation}
	x B_\chi\left(17\text{ Gen } r\right) \longrightarrow \text{Bew}_\chi\left[\text{Neg } Sb\left(r\substack{17\\ Z\left(x\right)}\right)\right]
\end{equation}

Daraus ergibt sich:\label{chibeweisbarkeit}

\begin{enumerate}
	\item $17 \text{ Gen } r$\mnote{$17 \text{ Gen } r \equiv \forall x_1: r$} ist nicht
		$\chi$-beweisbar\footnote{$x$ ist $\chi$-beweisbar, soll bedeuten:
		$x \epsilon \text{Flg}\left(x\right)$,
		was nach (7) dasselbe besagt wie: $\text{Bew}_\chi\left(x\right)$.}. Denn
		wäre dies der Fall, so gäbe es (nach 6.1) ein $n$, so daß
		$n B_\chi \left(17 \text{ Gen } r\right)$\mnote{$n$ ist ein Beweis
		in $\chi$ für $\left(17 \text{ Gen } r\right)$.}. Nach 17 gälte also:
		$\text{Bew}_\chi \left[\text{Neg }Sb\left(r\substack{17\\ Z\left(n\right)}\right)\right]$,
		während andererseits aus der $\chi$-\textit{Beweisbarkeit} von $17 \text{ Gen } r$ auch
		die von $Sb \left(r\substack{17\\ Z\left(n\right)}\right)$ folgt. $\chi$ wäre also\mnote{Daraus
		würde folgen, dass $\forall x_1: r$ sowohl beweisbar, als auch nicht-beweisbar wäre: ein Widerspruch.}
		widerspruchsvoll (umso mehr $\omega$-widerspruchsvoll).

	\item $\text{Neg }\left(17 \text{ Gen } r\right)$ ist nicht
		$\chi$-\textit{beweisbar}. Beweis: wie eben bewiesen wurde,
		ist $17 \text{ Gen } r$ nicht $\chi$-beweisbar, d.\,h.
		(nach 6.1) es gilt $\left(n\right)\overline{n B_\chi\left(17\text{ Gen } r\right)}$.
		Daraus folgt nach (15) $\left(n\right) \text{Bew}_\chi\left[Sb\left(r\substack{17\\ Z\left(n\right)}\right)\right]$,
		was zusammen mit $\text{Bew}_\chi\left[\text{Neg }\left(17 \text{ Gen } r\right)\right]$
		gegen die $\omega$-Widerspruchsfreiheit von $\chi$ verstoßen würde.

	$17\text{ Gen } r$ ist also aus $\chi$ unentscheidbar, womit Satz IV bewiesen ist.
\end{enumerate}

\let\originalfootnote=\thefootnote
\let\thefootnote=\fnfunfundvierziga
Man kann sich leicht überzeugen,\mnote{Jaja, \frq leicht\flq\dots}
daß der eben geführte Beweis konstruktiv ist\footnote{Denn alle im Beweise
vorkommenden Existentialbehauptungen beruhen auf Satz, der, wie leicht zu sehen,
intuitionistisch einwandfrei ist.}\mnote{\customlabel{konstruktivitaet}{\arabic{commentaryNumber}}\frq Konstruktivität\flq\ heißt 
in diesem Kontext: jeder, der genug Zeit und Mühe mitbringt, kann tatsächlich einen solchen
Widerspruch real nach den gegebenen Regeln konstruieren und faktisch vor sich haben.},
\let\thefootnote=\originalfootnote
\setcounter{footnote}{45}
d.\,h. es ist intuitionistisch einwandfrei folgendes bewiesen:
Sei eine beliebige rekursiv definierte Klasse $\chi$
von \textit{Formeln} vorgelegt. Wenn dann eine formale Entscheidung (aus $\chi$) für die (effektiv aufweisbare)
\textit{Satzformel} $17 \text{ Gen } r$ vorgelegt ist, so kann man effektive angeben:

\begin{enumerate}
	\item Einen \textit{Beweis} für $\text{Neg }\left(17 \text{ Gen } r\right)$.
	\item Für jedes beliebige $n$ einen Beweis für $Sb\left(r\substack{17\\ Z\left(n\right)}\right)$, d.\,h. eine formale Entscheidung von $17\text{ Gen } r$ würde die effektive Aufweisbarkeit eines $\omega$-Widerspruches zur Folge haben.
\end{enumerate}

Wir wollen eine Relation (Klasse) zwischen natürlichen Zahlen $R\left(x_1 \dots x_n\right)$
\so{entscheidungsdefinit} nennen, wenn es ein $n$-stelliges \textit{Relationszeichen} $r$ gibt,
so daß (3) und (4) (vgl. Satz V) gilt.
Insbesondere ist also nach Satz V jede rekursive Relation
entscheidungsdefinit. Analog soll ein \textit{Relationszeichen} entscheidungsdefinit
 heißen, wenn es auf diese Weise einer entscheidungsdefiniten Relation zugeordnet ist.
Es genügt nun für die Existenz unentscheidbarer Sätze, von der Klasse $\chi$ vorauszusetzen,
daß sie $\omega$-widerspruchsfrei und entscheidungsdefinit ist. Denn die
Entscheidungsdefinitheit überträgt sich von $\chi$ auf $x B_\chi y$ (vgl. (5), (6)) und
auf $Q\left(x, y\right)$ (vgl. (8.1)) und nur dies wurde im obigen Beweise verwendet.
Der unentscheidbare Satz hat in diesem Fall die Gestalt $v\text{ Gen } r$, wo $r$ ein
entscheidungsdefinites \textit{Klassenzeichen} ist (es genügt übrigens sogar,
daß $\chi$ in dem durch $\chi$ erweiterten System entscheidungsdefinit ist).

Setzt man von $\chi$ statt $\omega$-Widerspruchsfreiheit, bloß
Widerspruchsfreiheit voraus, so folgt zwar nicht die Existenz
 eines unentscheidbaren Satzes, wohl aber die Existenz einer
Eigenschaft $\left(r\right)$, für die weder ein Gegenbeispiel
\so{angebbar}, noch beweisbar ist, daß sie allen Zahlen zukommt.
Denn zum Beweise, dass $17\text{ Gen } r$ nicht
$\chi$-\textit{beweisbar} ist, wurde nur die Widerspruchsfreiheit
von $\chi$ verwendet (Vgl. S. \pageref{chibeweisbarkeit}) und aus
 $\overline{\text{Bew}_\chi}\left(17\text{ Gen } r\right)$ folgt
nach (15), daß für jede Zahl $x$
$Sb\left(r\substack{17\\ Z\left(x\right)}\right)$, folglich für
 keine Zahl $\text{Neg }Sb\left(r\substack{17\\ Z\left(x\right)}\right)$ $\chi$-\textit{beweisbar} ist.

Adjungiert man $\text{Neg }\left(17 \text{ Gen } r\right)$ zu $\chi$,
so erhält man eine widerspruchsfreie aber nicht $\omega$-widerspruchsfreie
\textit{Formelklasse} $\chi'$. $\chi'$ ist widerspruchsfrei, denn sonst wäre
$17 \text{ Gen } r$ $\chi$-\text{beweisbar}.
$\chi'$ ist aber nicht $\omega$-widerspruchsfrei, denn wegen
$\overline{\text{Bew}_\chi}\left(17\text{ Gen } r\right)$ und (15) gilt:
$\left(x\right) \text{Bew}_\chi Sb\left(r\substack{17\\ Z\left(x\right)}\right)$, umso mehr also:
$\left(x\right)\text{Bew}_{\chi'} Sb\left(r\substack{17\\ Z\left(x\right)}\right)$
und anderseits gilt natürlich:
$\text{Bew}_{\chi'}\left[\text{Neg }\left(17\text{ Gen }r\right)\right]$\footnote{\label{fussnote46}Die Existenz
widerspruchsfreie und nicht $\omega$-widerspruchsfreier $\chi$ ist damit natürlich nur
unter der Voraussetzung bewiesen, daß es überhaupt widerspruchsfreie
$\chi$ gibt (d.\,h. daß $P$ widerspruchsfrei ist).}.

Ein Spezialfall von Satz VI ist der, daß die Klasse $\chi$ aus endlich
vielen \textit{Formeln} (und ev. den daraus durch \textit{Typenerhöhung}
entstehenden) besteht. Jede endliche Klasse $\alpha$ ist natürlich rekursiv.
Sei $a$ die größte in $\alpha$ enthaltene Zahl\mnote{\frq die größte in $\alpha$ enthaltene Zahl\flq\ bedeutet:
der größte Primfaktor von $\alpha$.}. Dann gilt in diesem
Fall für $\chi$:

$$
x \epsilon \chi \thicksim \left(Em, n\right) \left[m \leqq x \& n \leqq a \& n \epsilon \alpha \& x = m Th\ n\right]
$$

$\chi$ ist also rekursiv. Das erlaubt z.\,B. zu schließen, daß auch mit Hilfe
des Auswahlaxioms (für alle Typen) oder der verallgemeinerten Kontinuumshypothese\mnote{Lapsig
gesagt ist die Kontinuumshypothese die Frage,
ob es zwischen der Dichten der Mengen $\mathbb{Q}$ und $\mathbb{R}$ noch eine
weitere Menge gibt.} nicht alle Sätze entscheidbar sind, vorausgesetzt, daß diese Hypothesen
$\omega$-widerspruchsfrei sind.

Beim Beweise von Satz VI wurden keine anderen Eigenschaften des Systems $P$ verwendet als die folgenden:

\begin{enumerate}
	\item Die Klassen der Axiome und die Schlußregeln
	 (d.\,h. die Relation \glqq unmittelbare Folge\grqq) sind rekursiv definierbar (sobald man die Grundzeichen in irgend einer Weise durch natürliche Zahlen ersetzt).
	 \item Jede rekursive Relation ist innerhalb des Systems $P$ definierbar (im Sinn von Satz V).
\end{enumerate}

Daher gibt es in jedem formalen System, das den Voraussetzungen 1, 2 genügt
und $\omega$-widerspruchsfrei ist, unentscheidbare Sätze der Form
$\left(x\right)F\left(x\right)$, wo $F$ eine rekursiv definierte Eigenschaft
natürlicher Zahlen ist, un ebenso in jeder Erweiterung eines solchen Systems
durch eine rekursiv definierbare $\omega$-widerspruchsfreie Klasse von Axiomen.
Zu den Systemen, welche die Voraussetzungen 1, 2 erfüllen, gehören, wie man leicht
bestätigen kann, das \so{Zermelo-Fraenkel}sche und das \so{v.~Neumann}sche Axiomensystem\mnote{Die von
Gödel aufgedeckte Problematik gilt für extrem viele logische Systeme, vgl. Kommentar \ref{bedingungen}.}
der Mengenlehre\footnote{Der Beweis von Voraussetzung 1. gestaltet sich hier sogar einfacher
als im Falle des Systems $P$, da es nur eine Art von Grundvariablen gibt (bzw. zwei bei J.~v.~Neumann).},
ferner das Axiomensystem der Zahlentheorie, welches aus
den \so{Peano}schen Axiomen, der rekursiven Definition [nach Schema (2)]
und den logischen Regeln besteht\footnote{Vgl. Problem III in D. \so{Hilbert}s Vortrag:
Probleme der Grundlegung der Mathematik. Math. Ann. 102.}.

\let\originalfootnote=\thefootnote
\let\thefootnote=\fnachtundvierziga
Die Voraussetzung 1. erfüllt überhaupt jedes System, dessen Schlußregeln die gewöhnlichen
sind und dessen Axiome (analog wie in $P$) durch Einsetzung aus endlich vielen Schemata
entstehen\footnote{Der wahre Grund für die
Unvollständigkeit, welche allen formalen Systemen
der Mathematik liegt anhaftet, liegt, wie im II.
Teil dieser Abhandlung gezeigt werden wird, darin,
daß die Bildung immer höherer Typen sich ins
Transfinite fortsetzen läßt. (Vgl. D.~\so{Hilbert},
Über das Unendliche, Math. An. 95. S. 184), während
in jedem formalen System höchstens abzählbar viele
vorhanden sind. Man kann nämlich zeigen, daß die hier
aufgestellten unentscheidbaren Sätze durch Adjunktion
passender höherer Typen (z.\,B. des Typus $\omega$ zum
System $P$) immer entscheidbar werden. Analoges gilt auch
für das Axiomensystem der Mengenlehre.}.
\let\thefootnote=\originalfootnote
\setcounter{footnote}{48}

\begin{center}
3.
\end{center}

Wir ziehen nun aus Satz VI weitere Folgerungen und geben zu diesem Zweck folgende Definition:

Eine Relation (Klasse) heißt arithmetisch, wenn sie allein
mittels der Begriffe $+$, $.$\mnote{D.\,h. sie
zurückführbar ist auf einfach Integer-Arithmetik und einfache
logische Aussagen.} [Addition und Multiplikation, bezogen auf natürliche
Zahlen\footnote{Die Null wird hier und im folgenden immer
mit zu den natürlichen Zahlen gerechnet.}] und den
logischen Konstanten $\lor, \overline{\phantom{XX}}, \left(x\right), =$ definieren
lässt, wobei $\left(x\right)$ und $=$ sich nur auf natürliche
Zahlen beziehen dürfen\footnote{Das Definiens eines solchen
Begriffes muß sich also allein mittels der angeführten Zeichen,
Variablen für natürliche Zahlen $x, y, \dots$ und den Zeichen
$0, 1$ aufbauen (Funktions- und Mengenvariable dürfen nicht vorkommen).
(In den Präfixen darf statt $x$ natürlich auch jede andere Zahlvariable
stehen.)}. Entsprechend wird der Begriff \glqq arithmetischer Satz\grqq\ definiert.
Insbesondere sind z.\,B. die Relation \glqq größer\grqq\ und \glqq kongruent
nach einem Modul\grqq\ arithmetisch, denn es gilt:

$$
x > y \thicksim \overline{\left(Ez\right)}\left[y = x + z\right]
$$\mnote{Wenn $x$ größer $y$ ist, existiert kein $z \in \mathbb{N}$, so dass $y = x + z$.}

$$
x \equiv y \left(\text{mod} n\right) \thicksim \left(Ez\right) \left[x = y + z.n \lor y = x + z.n\right]
$$

Es gilt der

Satz VII: \so{Jede rekursive Relation ist arithmetisch}.

Wir beweisen den Satz in der Gestalt: Jede Relation der Form
$x_n = \phi\left(x_1\dots x_n\right)$, wo $\phi$rekursiv ist, ist arithmetisch, und wenden die vollständige Induktion nach der Stufe von $\phi$ an. $\phi$ habe die $s$-te Stufe ($s > 1$). Dann gilt entweder:

$$
1. \phi\left(x_1 \dots x_n\right) = \rho\left[\chi_1\left(x_1 \dots x_n\right), \chi_2\left(x_1 \dots x_n\right) \dots \chi_m \left(x_1 \dots x_n\right)\right]\footnote{Es brauchen natürlich nicht alle $x_1 \dots x_n$ in den $\chi_i$ tatsächlich vorkommen [vgl. das Beispiel in Fußnote $^{27}$].}
$$
(wo $\rho$ und sämtliche $\chi_i$ kleinere Stufen haben als $s$) oder:

$$
2. \phi\left(0, x_2 \dots x_n\right) = \psi\left(x_2\dots x_n\right)
$$

$$
\phi\left(k + 1, x_2 \dots x_n\right) = \mu\left[k, \phi\left(k, x_2 \dots x_n\right), x_2 \dots x_n\right]
$$
(wo $\psi, \mu$ niedrigere Stufen als $s$ haben) oder:

$$
2. \phi\left(0, x_2 \dots x_n\right) = \psi\left(x_2 \dots x_n\right)
$$

$$
\phi\left(k + 1, x_2 \dots x_n\right) = \mu\left[k, \phi\left(k, x_2 \dots x_n\right), x_2 \dots x_n\right]
$$

(wo $\psi, \mu$ niedrigere Stufen als $s$ haben).

Im ersten Falle gilt:

$$
x_0 = \phi\left(x_1 \dots x_n\right) \thicksim \left(Ey_1 \dots y_m\right) \left[R\left(x_0 y_1 \dots y_m\right) \& \right.
$$\\[\spacebetweenbreakedequations]
$$\left.S_1\left(y_1, x_1 \dots x_n\right) \& \dots \& S_m\left(y_m, x_1 \dots x_n\right)\right],
$$\\[\spaceafterbreakedequation]

wo $R$ bzw. $S_i$ die nach induktiver Annahme existierenden mit $x_0 = \rho\left(y_1 \dots y_m\right)$ bzw.
$y = \chi_i\left(x_1 \dots x_n\right)$ äquivalenten arithmetischen
Relationen sind. Daher ist $x_0 = \phi\left(x_1\dots x_n\right)$ in
diesem Fall arithmetisch.

Im zweiten Fall wenden wir folgendes Verfahren an: Man
kann die Relation $x_0 = \phi\left(x_1 \dots x_n\right)$ mit Hilfe des Begriffes \glqq Folge von Zahlen\grqq\
($f$)\footnote{$f$ bedeutet hier eine Variable, deren Wertbereich die Folgen natürl. Zahlen sind. Mit $f_k$ wird das $k + 1$-te Glied einer Folge $f$ bezeichnet (mit $f_0$ das Erste).}
folgendermaßen ausdrücken:

$$
x_0 = \phi\left(x_1 \dots x_n\right) \thicksim \left(Ef\right)\left\{ f_0 = \psi\left(x_2 \dots x_n\right) \& \left(k\right) \left[k < x_1 \longrightarrow\right.\right.
$$\\[\spacebetweenbreakedequations]
$$\left.\left. f_{k + 1} = \mu\left(k, f_k, x_2 \dots x_n\right)\right] \& x_0 = f_{x_1}\right\}
$$\\[\spaceafterbreakedequation]

Wenn $S\left(y, x_2 \dots x_n\right)$ bzw. $T\left(z, x_1 \dots x_{n + 1}\right)$ die
nach induktiver Annahme existieren mit $y = \psi\left(x_2 \dots x_n\right)$ bzw.
$z = \mu\left(x_1 \dots x_{n + 1}\right)$ äquivalenten arithmetische Relationen sind, gilt daher:

\begin{equation}
x_0 = \phi\left(x_1 \dots x_n\right) \thicksim \left(Ef\right) \left\{S\left(f_0, x_2 \dots x_n\right) \& \left(k\right)\left[k < x_1 \longrightarrow \right.\right.
\end{equation}\\[\spacebetweenbreakedequations]
\begin{equation*}
\left.\left.T\left(f_{k + 1}, k, f_k, x_2 \dots x_n\right)\right] \& x_0 = f_{x_1}\right\}
\end{equation*}\\[\spaceafterbreakedequation]

Nun ersetzen wir den Begriff \glqq Folge von Zahlen\grqq\ durch
\glqq Paar von Zahlen\grqq, indem wir dem Zahlenpaar $n, d$
die Zahlenfolge $f^{\left(n, d\right)}$ $\left(f_k^{n, d} = \left[n\right]_{1 + \left(k + 1\right) d})\right)$
zuordnen, wobei $\left[n\right]_p$ den kleinsten nicht negativen Rest von $n$ modulo $p$ bedeutet.

Es gilt dann der

Hilfssatz 1: ist $f$ eine beliebige Folge natürlicher
Zahlen und $k$ eine beliebige natürliche Zahl, so gibt es
ein paar von natürlichen Zahlen $n, d$, so daß $f^{\left(n, d\right)}$
und $f$ in den ersten $k$ Gliedern übereinstimmen.

Beweis: Sei $l$ die größte der Zahlen $k, f_0, f_1 \dots f_{k - 1}$. Man bestimme $n$ so, daß:

$$
n \equiv \left[\text{mod }\left(1 + \left(i + 1\right) l!\right)\right]\text{ für } i = 0, 1 \dots k - 1
		$$
		was möglich ist, da jede der Zahlen $1 + \left(i + 1\right) l!$ $\left(i = 0, 1 \dots k - 1\right)$
		relativ prim sind. Denn eine in den zwei von diesen Zahlen enthaltene Primzahl
		müßte auch in der Differenz $i_1 - i_2)l!$ und daher wegen
$\left|i_1 - i_2\right| < l$ in $l!$ enthalten sein, was unmöglich
ist. Das Zahlenpaar $n, l!$ leistet dann das Verlangte.

Da die Relation $x = \left[n\right]_p$ durch:

$$
x \equiv n \left(\text{mod } p\right) \& x < p
$$

definiert und daher arithmetisch ist, so ist auch die folgendermaßen definierte Relation $P\left(x_0, x_1 \dots x_n\right)$:

$$
P\left(x_0 \dots x_n\right) \equiv \left(En, d\right) \left\{S\left(\left[n\right]_{d + 1}, x_2 \dots x_n\right) \& \left(\right) \left[k < x_1 \longrightarrow\right.\right.
$$\\[\spacebetweenbreakedequations]
$$ \left.\left.T\left(\left[n\right]_{1 + 1 \left(k + 2\right)}, k, \left[n\right]_{1 + d \left(k + 1\right)}, x_2 \dots x_n\right)\right] \& x_0 = \left[n\right]_{1 + d \left(x_1 + 1\right)}\right\}
$$\\[\spaceafterbreakedequation]
arithmetisch, welche nach (17) und Hilfssatz 1 mit:
$x_0 = \phi\left(x_1 \dots x_n\right)$ äquivalent ist (es kommt
		bei der Folge $f$ in (17) nur auf ihren Verlauf
		bis zum $x_1 + 1$-ten Glied an). Damit ist der
Satz~VII bewiesen.

Gemäß Satz VII gibt es zu jedem Problem der Form
$\left(x\right)F\left(x\right)$ ($F$ rekursiv) ein äquivalentes arithmetisches
Problem und da der ganze Beweis von Satz VII sich (für jedes spezielle $F$) innerhalb des Systems $P$ formalisieren läßt, ist diese Äquivalenz in $P$ beweisbar. Daher gilt:

\label{satzvii}\so{Satz VIII: In jedem der in Satz VI genannten formalen Systeme\footnote{Das sind diejenigen $\omega$-widerspruchsfreien Systeme, welche aus $P$ durch Hinzufügung einer rekursiv definierbaren Klasse von Axiomen entstehen.}
gibt es unentscheidbare arithmetische Sätze.}

Dasselbe gilt (nach der Bemerkung auf Seite \pageref{fussnote46})
für das Axiomensystem der Mengenlehre und dessen Erweiterungen durch $\omega$-widerspruchsfreie rekursive
Klassen von Axiomen.

Wir leiten schließlich noch folgendes Resultat her:

\so{Satz IX: In allen in Satz VI genannten formalen Systeme gibt es
unentscheidbare Probleme des engeren Funktionenkalküls}\footnote{Vgl. \so{Hilbert-Ackermann},
Grundzüge der theoretischen Logik. Im System $P$ sind unter Formeln des engeren
Funktionenkalküls diejenigen zu verstehen, welche aus den Formeln des engeren
Funktionenkalküls der PM durch die auf S. \pageref{ersetzungdurchklassen} angedeutete Ersetzung der
Relationen durch Klassen höheren Typs entstehen.} (d.\,h. Formeln des engeren
Funktionenkalküls, für die weder Allgemeingültigkeit noch Existenz eines
Gegenbeispiels beweisbar ist)\footnote{In meiner Arbeit: \citefield{goedelvollstaendigkeit}{note},
habe ich gezeigt, daß jede Formel des engeren Funktionenkalküls entweder als
allgemeingültig nachweisbar ist oder ein Gegenbeispiel existiert; die Existenz
dieses Gegenbeispiels ist aber nach Satz IX \so{nicht} immer nachweisbar (in den
angeführten formalen Systemen).}.

Dies beruht auf:

\so{Satz X: Jedes Problem der Form $\left(x\right)F\left(x\right)$ ($F$ rekursiv)
läßt sich zurückführen auf die Frage nach der Erfüllbarkeit einer Formel des engeren
Funktionenkalküls} (d.\,h. zu jedem rekursiven $F$ kann man eine Formel des engeren
Funktionenkalküls angeben, deren Erfüllbarkeit mit der Richtigkeit von
$\left(x\right)F\left(x\right)$ äquivalent ist).

Zum engeren Funktionenkalkül (e. F.) rechnen wir diejenigen Formeln, welche sich aus den Grundzeichen
$\overline{\phantom{XX}}, \lor, \left(x\right), =, x, y \dots \text{(Individuenvariable)} F\left(x\right), G\left(x y\right), H\left(x, y, z\right) \dots $
(Eigenschafts- und Relationsvariable) aufbauen\footnote{D.~\so{Hilbert} und \so{W.~Ackermann} rechnen
in dem eben zitierten Buch das Zeichen $=$ nicht zum engeren Funktionenkalkül. Es geht aber zu jeder
Formel in der das Zeichen $=$ vorkommt, eine solche ohne dieses Zeichen, die mit der ursprünglichen
gleichzeitig erfüllbar ist (vgl. die in Fußnote $^{55}$ zitiert Arbeit).},
wobei $\left(x\right)$ und $=$ sich nur
 auf Individuen beziehen dürfen. Wir fügen zu
 diesen Zeichen noch eine dritte Art von Variablen
$\phi\left(x\right), \psi\left(x y\right), \chi\left(x y z\right)$ hinzu,
 die Gegenstandsfunktionen vertreten (d.\,h. $\phi\left(x\right), \psi\left(x y\right)$
 etc. bezeichnen eindeutige Funktionen, deren Argumente und Werte Individuen sind\footnote{Und zwar soll der Definitionsbereich immer der \so{ganze} Individuenbereich sein.}.
Eine Formel, die außer den zuerst angeführten Zeichen
des e.~F. noch Variable dritter Art ($\phi\left(x\right), \psi\left(x y\right)$ etc.)
enthält, soll eine Formel im weiteren Sinne (i.~w.~S.) heißen\footnote{Variable dritter Art dürfen dabei an
allen Leerstellen für Individuenvariable stehen, z.\,B.:
$y = \phi\left(x\right), F\left(x, \phi\left(y\right)\right), G\left[\psi\left(x, \phi\left(y\right)\right), x\right]$
usw.}.
Die Begriffe \glqq erfüllbar\grqq, \glqq allgemeingültig\grqq\ übertragen
sich ohneweiters auf die Formel i.~w.~S. und es gilt der Satz, daß man zu
jeder Formel i.~w.~S. $A$ eine gewöhnliche Formel des e.~F.
$B$ angeben kann, so daß die Erfüllbarkeit von $A$ mit der von $B$ äquivalent
ist. $B$ erhält man aus $A$, indem man die in $A$ vorkommenden Variablen dritter
Art $\phi\left(x\right), \psi\left(x y\right) \dots$ durch Ausdrücke der Form
$\left(\i z\right)F\left(z x\right), \left(\i z\right)G\left(z, x y\right)\dots$ ersetzt,
die \glqq beschreibenden\grqq\ Funktionen im Sinne der PM I $\pmstar$ 14
auflöst und die so erhaltene Formel mit einem Ausdruck logisch
multipliziert\footnote{D.\,h. die Konjunktion bildet.},
der besagt, daß sämtliche an Stelle der $\phi, \psi \dots$ gesetzte
$F, G\dots$ hinsichtlich der ersten Leerstelle genau eindeutig sind.

Wir zeigen nun, daß es zu jedem Problem der Form $\left(x\right)F\left(x\right)$ ($F$ rekursiv) ein äquivalentes betreffend die Erfüllbarkeit einer Formel i.~w.~S. gibt, woraus nach der eben gemachten Bemerkung Satz X folgt.

Da $F$ rekursiv ist, gibt es eine rekursive Funktion
$\Phi\left(x\right)$, so daß $F\left(x\right) \thicksim \left[\Phi\left(x\right) = 0\right]$,
und für $\Phi$ gibt es eine Reihe von Funktionen $\Phi_1, \Phi2 \dots \Phi_n$,
so daß: $\Phi_n = \Phi, \Phi_1\left(x\right) = x + 1$ und für jedes $\Phi_k\left(1 < k \leqq n\right)$ entweder:

\begin{equation}
1. \left(x_2 \dots x_m\right) \left[\Phi_k\left(0, x_2 \dots x_m\right) = \Phi_p\left(x_2 \dots x_m\right)\right]
\end{equation}
\begin{equation*}
\left(x, x_2 \dots x_m\right) \left\{\Phi_k\left[\Phi_1\left(x\right), x_2 \dots x_m\right] = \Phi_q \left[x, \Phi_k\left(x, x_2 \dots x_m\right), x_2 \dots x_m\right]\right\}
\end{equation*}
\begin{equation*}
p, q < k
\end{equation*}

oder:

\begin{equation}
2. \left(x_1 \dots x_m\right)%
\left[
	\Phi_k\left(x_1 \dots x_m\right) =%
	\Phi_r\left(%
		\Phi_{i_1}\left(\mathfrak{x_1}\right) \dots%
	\Phi_{i_1}\left(\mathfrak{x}_s%
	\right)%
	\right)%
\right]%
\footnote{$\mathfrak{x}_i \left(i = 1 \dots s\right)$ vertreten irgend welche Komplexe der Variablen $x_1, x_2 \dots x_m$, z.\,B. $x_1 x_3 x_2$.}
\end{equation}
\begin{equation*}
r < k, i_v < k \left(\text{für } v = 1, 2 \dots s\right)
\end{equation*}

oder:

\begin{equation}
3. \left(x_1 \dots x_m\right) \left[\Phi_k\left(x_1 \dots x_m\right) = \Phi_1\left(\Phi_1 \dots \Phi_1\left(0\right)\right)\right]
\end{equation}

Ferner bilden wir die Sätze:

\begin{equation}
\left(x\right) \overline{\Phi_1\left(x\right) = 0} \& \left(x y\right) \left[\Phi_1\left(x\right) = \Phi_1\left(y\right) \longrightarrow x = y\right]
\end{equation}

\begin{equation}
\left(x\right)\left[\Phi_n\left(x\right) = 0\right]
\end{equation}

Wir ersetzen nun in allen Formeln (18), (19), (20) (für $k = 2, 3 \dots n$) und in (21) (22) die Funktionen $\Phi_i$ durch Funktionsvariable
$\phi_i$, die Zahl $0$ durch eine sonst nicht vorkommende
Individuenvariablen $x_0$ und bilden die Konjunktion $C$ sämtlicher so erhaltener Formeln.

Die Formel $\left(Ex_0\right) C$ hat dann die verlangte Eigenschaft, d.\,h.

\begin{enumerate}
	\item Wenn $\left(x\right)\left[\Phi\left(x\right) = 0\right]$ gilt, ist $\left(Ex_0\right)C$ erfüllbar, denn die Funktionen $\Phi_1, \Phi_2 \dots \Phi_n$ ergeben dann offenbar in $\left(Ex_0\right)C$ für $\phi_1, \phi2 \dots \phi_n$ eingesetzt einen richtigen Satz.
	\item Wenn $\left(Ex_0\right)C$ erfüllbar ist, gilt $\left(x\right)\left[\Phi\left(x\right) = 0\right]$.
\end{enumerate}

Beweis: Seien $\Psi_1, \Psi_2 \dots \Psi_n$ die nach Voraussetzung
existierende Funktionen, welche in $\left(Ex_0\right)C$ für
$\phi_1, \phi_2 \dots \phi_n$ eingesetzt einen richtigen Satz liefern.
Ihr Individuenbereich Sei $\mathfrak{J}$. Wegen der Richtigkeit von
$\left(Ex_0\right)C$ für die Funktion $\Psi_4$ gibt es ein Individuum $a$
(aus $\mathfrak{J}$), so, daß sämtliche Formeln (18) bis (22)
bei der Ersetzung der $\Phi_i$ durch $\Psi_i$ und von 0 durch $a$
in richtige Sätze ($18'$) bis ($22'$) übergeben. Wir bilden nun
die kleinste Teilklasse von $\mathfrak{J}$, welche $a$ enthält
und gegen die Operation $\Psi_1\left(x\right)$ abgeschlossen ist.
Diese Teilklasse ($\mathfrak{J}'$) hat die Eigenschaft, daß
jede der Funktionen $\Psi_i$ auf Elemente aus $\mathfrak{J}'$
angewendet wieder Elemente aus $\mathfrak{J}'$ ergibt.
Denn für $\Psi_1$ gilt dies nach Definition von $\mathfrak{J}'$
und wegen ($18'$), ($19'$), ($20'$) überträgt sich diese
Eigenschaft von $\Psi_i$ mit niedrigerem Index auf solche mit höherem.
Die Funktion, welche aus $\Psi_i$ durch Beschränkung auf den
Individuenbereich $\mathfrak{J}'$ entstehen, nennen wir
$\Psi_i'$. Auch für diese Funktion gelten sämtliche Formeln
(18) bis (22) (bei der  Ersetzung von 0 durch $a$ und $\Phi_i$ durch $\Psi_i'$).

Wegen der Richtigkeit von (21) für $\Psi_1'$ und $a$ kann man die Individuen aus $\mathfrak{J}'$ eineindeutig auf die natürlichen Zahlen abbilden u. zw. so, daß $a$ in $0$ und die Funktion $\Psi_1'$ in die Nachfolgerfunktion $\Phi_1$ übergeht. Durch diese Abbildung gehen aber sämtliche Funktionen $\Psi_i'$ in die Funktionen $\Phi_i$ über und wegen der Richtigkeit von (22) für $\Psi_n'$ und $a$ gilt $\left(x\right)\left[\Phi_n\left(x\right) = 0\right]$ oder $\left(x\right)\left[\Phi\left(x\right) = 0\right]$, was zu beweisen war\footnote{Aus Satz X folgt z.\,B., daß das \so{Fermat}sche und das \so{Goldbach}sche Problem lösbar wären, wenn man das Entscheidungsproblem des e.~F. gelöst hätte.}.

Da man die Überlegungen, welche zu Satz X führen, (für jedes spezielle $F$) auch innerhalb des Systems $P$ durchführen kann, so ist die Äquivalenz zwischen einem Satz der Form $\left(x\right)F\left(x\right)$ ($F$ rekursiv) und der Erfüllbarkeit der entsprechenden Formel des e.~F. in $P$ beweisbar und daher folgt aus der Unentscheidbarkeit des einen die des anderen, womit Satz IX bewiesen ist.\footnote{Satz IX gilt natürlich auch für das Axiomensystem der Mengenlehre und dessen Erweiterungen durch rekursiv definierbar $\omega$-widerspruchsfreie Klassen von Axiomen, da es ja auch in diesem Systemen unentscheidbare Sätze der Form $\left(x\right)F\left(x\right)$ ($F$ rekursiv) gibt}.

\begin{center}
4.
\end{center}

Aus den Ergebnissen von Abschnitt 2 folgt ein merkwürdiges Resultat, bezüglich eines Widerspruchslosigkeitsbeweises des Systems $P$ (und seiner Erweiterungen), der durch folgenden Satz ausgesprochen wird:

\so{Satz XI: Sei $\chi$ eine beliebige rekursive
widerspruchsfreie\footnote{$\chi$ ist widerspruchsfrei
(abgekürzt als $\text{Wid }\left(x\right)$) wird folgendermaßen
definiert:
$\text{Wid}\left(x\right)\equiv\left(Ex\right)\left[\text{Form}\left(x\right)\&\text{Bew}_\chi\left(x\right)\right]$.}
Klasse von}\label{satzxi} \textit{Formeln}\so{, dann gilt: Die }\textit{Satzformel}\so{, welche besagt,
daß $\chi$ widerspruchsfrei ist, ist nicht} $\chi$-\textit{beweisbar}; insbesondere ist\mnote{Zweiter
Unvollständigkeitssatz: ein System kann aus sich selbst heraus seine eigene Widerspruchsfreiheit nicht
beweisen.} die Widerspruchsfreiheit von $P$ in $P$ unbeweisbar\footnote{Dies folgt, wenn man für
$\chi$ die leere Klasse von \textit{Formeln} einsetzt.},
vorausgesetzt, daß $P$ widerspruchsfrei ist (im entgegengesetzten Fall ist natürlich jede Aussage beweisbar).

Der Beweis ist (in Umrissen skizziert) der folgende: Sei $\chi$ eine beliebige für die folgenden
Betrachtungen ein für allemal gewählte rekursive Klasse von \textit{Formeln} (im einfachsten Falle die leere Klasse).
Zum Beweise der Tatsache, daß $17\text{ Gen }r$ nicht $\chi$-\textit{beweisbar} ist\footnote{$r$
hängt natürlich (ebenso wie $p$) von $\chi$ ab.}, wurde, wie aus 1. Seite \pageref{chibeweisbarkeit} hervorgeht,
nur die Widerspruchsfreiheit von $\chi$ benutzt, d.\,h. es gilt:

\begin{equation}
\text{Wid }\left(x\right) \longrightarrow \overline{\text{Bew}_x} \left(17\text{ Gen } r\right)
\end{equation}
d.\,h. nach (6.1):

\begin{equation*}
\text{Wid }\left(x\right) \longrightarrow \left(x\right)\overline{x B_x \left(17\text{ Gen } r\right)}
\end{equation*}
Nach (13) ist $17\text{ Gen } r = Sb\left(p\substack{19\\ Z\left(p\right)}\right)$ und daher:

\begin{equation*}
\text{Wid}\left(x\right) \longrightarrow\left(x\right)\overline{x B_x Sb\left(p\substack{19\\ Z\left(p\right)}\right)}
\end{equation*}
d.\,h. nach (8.1):

\begin{equation}
\text{Wid}\left(x\right) \longrightarrow \left(x\right) Q\left(x, p\right)
\end{equation}

Wir stellen nun folgendes fest: Sämtliche in Abschnitt 2\footnote{Von der Definition für
\glqq rekursiv\grqq\ auf Seite \pageref{zwischenbetrachtungrekursion} bis zum Beweis von Satz VI inkl.}
und Abschnitt 4 bisher definierten Begriffe (bzw. bewiesene Behauptungen) sind auch in
$P$ ausdrückbar (bzw. beweisbar). Denn es wurden überall nur die gewöhnlichen Definitions-\mnote{Man braucht
keine extraordinäre \frq Magie\flq, um diese Sätze herzuleiten; es reichen die einfachen Grundlagen
der logischen Systeme.}
und Beweismethoden der klassischen Mathematik verwendet, wie sie im System $P$ formalisiert
sind. Insbesondere ist $\chi$ (wie jede rekursive Klasse) in $P$ definierbar.
Sei $w$ die \textit{Satzformel}, durch welche in $P\text{ Wid } \left(x\right)$
ausgedrückt wird. Die Relation $Q\left(x, y\right)$ wird gemäß (8.1), (9), (10)
durch das \textit{Relationszeichen} $q$ ausgedrückt, folglich
$Q\left(x, p\right)$ durch $r \left[\text{ da nach (12) } r = Sb\left(q\substack{19\\ Z\left(p\right)}\right)\right]$
und der Satz $\left(x\right)Q\left(x, p\right)$ durch $17\text{ Gen } r$.

Wegen (24) ist also $w\text{ Imp } \left(17\text{ Gen }r\right)$
in $P$ beweisbar\footnote{Daß aus (23) auf die Richtigkeit von
$w\text{ Imp }\left(17\text{ Gen } r\right)$
geschlossen werden kann, beruht einfach darauf, daß der unentscheidbare Satz %\mnote{$w \supset \left(\glqq 17\grqq\text{ Gen } r\right)$} % Todo was ist 17? Gen -> Satz 17, was ist r? Was ist w?
$17\text{ Gen }r$, wie gleich zu Anfang bemerkt, seine eigene Unbeweisbarkeit behauptet.}
(umso mehr $\chi$-\textit{beweisbar}). Wäre nun $w$ $\chi$-\textit{beweisbar},
so wäre auch $17\text{ Gen }r$ $\chi$-\textit{beweisbar} und daraus würde nach (23)
folgen, daß $\chi$ nicht widerspruchsfrei ist.

Es sei bemerkt, daß auch dieser Beweis konstruktiv\mnote{Vgl. Kommentar \ref{konstruktivitaet}.} ist, d.\,h. er 
gestattet, falls ein \textit{Beweis} aus $\chi$ für $w$ vorgelegt ist, ein Widerspruch aus $\chi$ effektiv 
herzuleiten. Der ganze Beweis für Satz XI läßt sich wörtlich auch auf das Axiomensystem 
der Mengenlehre $M$ und der klassischen 
Mathematik\footnote{Vgl. \citefield{hilbertschebeweistheorie}{note}.}
$A$ übertragen und liefert auch hier das Resultat: es gibt keinen
Widerspruchlosigkeitsbeweis\mnote{\customlabel{zweiterunvollstaendigkeitssatz}{\arabic{commentaryNumber}}Der zweite gödel'sche
Unvollständigkeitssatz; ein ausreichend komplexes System kann seine eigene Vollständigkeit
nicht beweisen.} für $M$ bzw. $A$, der innerhalb von $M$ bzw. $A$ formalisiert werden könnte, vorausgesetzt,
daß $M$ bzw. $A$ widerspruchsfrei ist. Es sei ausdrücklich bemerkt, daß Satz XI
(und die entsprechenden Resultate über $M, A$) in keinem Widerspruch zum \so{Hilbert}schen
formalistischen Standpunkt stehen. Denn dieser setzt nur die Existenz eines mit finiten
Mitteln geführten Widerspruchsfreiheitsbeweises voraus und es wäre denkbar, daß es finite
Beweise gibt, die sich in $P$ (bzw. $M, A$) \so{nicht} darstellen lassen.

Da es für jede widerspruchsfreie Klasse $\chi$ $w$ nicht
$\chi$-\textit{beweisbar} ist, so gibt es schon immer dann
(aus $\chi$) unentscheidbare Sätze (nämlich $w$, wenn
$\text{Neg }\left(w\right)$ nicht $\chi$-beweisbar ist;
m.~a.~W. kann man in Satz VI die Voraussetzung der
$\omega$-Widerspruchsfreiheit ersetzen durch die folgende:
Die Aussage \glqq $\chi$ ist widerspruchsvoll\grqq\ ist
nicht $\chi$-beweisbar. (Man beachte, daß es widerspruchsfreie
$\chi$ gibt, für die diese Aussage $\chi$-beweisbar ist.)

Wir haben uns in dieser Arbeit im wesentlichen auf das System $P$
beschränkt und die Anwendungen auf andere Systeme nur angedeutet.
In voller Allgemeinheit werden die Resultate in einer demnächst
erscheinenden Fortsetzung ausgesprochen und bewiesen werden.
In dieser Arbeit wird auch der nur skizzenhaft geführte Beweis
von Satz XI ausführlich dargestellt werden.

\begin{center}
(Eingelangt: 17.XI.1930)
\end{center}
\vspace{0.5cm}
\begin{center}
\rule{2cm}{0.01cm}
\end{center}

\newpage

\printbibliography[keyword={goedel}]

\end{document}

%  What Gödel did was create a mathematical statement that said, "The statement with code number 42 cannot be proved." But - and this is the clever bit - he did this in such a way that the statement with code number 42 was that very same statement. In short, it was really saying, "This sentence cannot be proved." So if it could be proved, it would be false; and if you can prove anything false, you can prove 1=0. But if it couldn't be proved, then it would be true. So, contrary to what we were all told in school, there would indeed be statements in math that could not be proved right or wrong, but which would in fact be right - you just couldn't prove it!
% -> Metamathematische Überlegungen haben das bewiesen, was im System unbeweisbar war
