\section*{Einleitung}

\let\originalthefootnote\thefootnote
\renewcommand*{\thefootnote}{\fnsymbol{footnote}}

\subsection*{Algorithmen}

\lettrine[nindent=0em]{\color{purple}A}{lgorithmen} sind endliche Abfolgen von\mnote{Für
die Spezifikationen der Schritte von Algorithmen siehe Knuth, TAOCP 1, S. 4f. Die Punkte sind:
\begin{itemize}
	\item Finiteness
	\item Defineteness
	\item Input
	\item Output
	\item (Effectiveness; eingeklammert, weil Knuth Computer und praktische
		Algorithmen im Kopf hat; Gödels Beweise aufzuschreiben wäre alles andere als effizient; es ist
		aber \textit{prinzipiell} möglich.)
\end{itemize}}
genau spezifizierten Schritten, um ein mathematisches oder logisches Problem zu lösen. 
Die Frage, vor der \so{Hilbert} stand, lautete: wenn man die Mathematik formalisiert, 
sind dann alle wahren Aussagen durch einen Algorithmus beweisbar? Oder: kann man durch
einen Algorithmus, der alle möglichen Aussagen durchschreitet, von jeder Einzelnen
in endlicher Zeit herausfinden, ob sie wahr ist oder nicht?

Für \so{Hilbert} schien die Sache klar zu sein: es musste so sein. Nur fehlte ihm der Beweis
dafür. Erst Gödel zeigte, dass die Sache viel vertrackter ist und so nicht geht.

\subsection*{Beweisbarkeit und Gödelnummern}

Was heißt \frq beweisbar\flq? Beweise sind Zurückführungen auf einzelne Axiome mithilfe
von syntaktischen Veränderungsregeln.\mnote{Zumindest lassen sie sich analog zu
solchen Regeln behandeln, wie die moderne Computertechnik eindrucksvoll zeigt. Vgl. Kommentar \ref{tractatus}. Vgl.
auch \url{https://www.youtube.com/watch?v=qT8NyyRgLDQ} für ein Weiterdenken in eine alternative Richtung, die
aber auch das \frq Gödelproblem\flq\ hat.} 
So kann man jede mathematische Aussage als Zeichenkette
auffassen und schauen, ob man -- durch erlaubte Veränderungen der Axiome des Systems --
irgendwie auf diese Zeichenkette kommt. Hierbei ist Gödel ein großer Vorreiter der
modernen Informatik, denn Computer können auch nur nach fest definierten Regeln
Zeichenketten bearbeiten. Die mathematische Formel wird dabei nur sekundär 
inhaltlich\mnote{Die inhaltliche Betrachtung findet erst
beim Menschen statt, der versucht, das, was die Formel aussagt, zu verstehen. Der Formel selbst
liegt diese inhaltliche Deutung nicht bei, was unter Anderem zur extrem breiten Verwendbarkeit
der Mathematik folgt. Vereinfacht gesagt: mit der gleichen Formel kann man berechnen, wie ein Atomkraftwerk funktioniert,
aber auch eine Atombombe; zwei sehr unterschiedliche Anwendungsbereiche mit ähnlicher mathematische Struktur.}
betrachtet und primär als Ansammlung von Zeichen, die nach Regeln bearbeitet werden.
Jede bewiesene mathematische Aussage lässt sich prinzipiell als sehr lange Zeichenkette
darstellen, die nur aus den Grundoperationen besteht (z.\,B. XOR, AND und NOT in
der modernen Computertechnik). Daraus lassen sich Operationen wie das Addieren 
herleiten, woraus sich das Multiplizieren herleiten lässt (mit ganzen Zahlen) usw.
Aber das sind alles nur Vereinfachungen, damit wir als Menschen, die damit arbeiten,
besser damit klarkommen. Inhaltlich ist ein $a + b$ gleichbedeutend mit einer langen
Kette an UND, XOR und NICHT-Operationen im CPU eines Computers.

Gödel zeigte, dass es nicht möglich ist, alle wahren Aussagen zu beweisen, 
in dem er einen Algorithmus anbietet, der -- komplett zurückführbar auf die logischen 
Grundaussagen --, immer Sätze konstruieren kann, die zwar durch metamathematische
Überlegungen von \frq außerhalb\flq\ des Systems\mnote{Vgl. Kommentar \ref{metamatematischeueberlegungen}.}
als geltend gezeigt werden können, die aber nicht innerhalb des Systems beweisbar sind.

\subsection*{Die \textit{Principia Mathematica} und die Typentheorie}

Russell und Whitehead haben erkannt, dass man solche Sätze mit selbstreferenziellen
Aussagen generieren kann, z.\,B. \frq dieser Satz ist falsch\flq,\mnote{\customlabel{luegezwei}{\arabic{commentaryNumber}} 
Kommentar \ref{luegeeins} ist falsch.} und daher haben sie
in der Principia Mathematica solche Sätze, die sich auf sich selbst beziehen, 
\frq verboten\flq\mnote{$\longrightarrow$ Typentheorie%TODO
}. Aber selbst, wenn man das verbietet, lassen sich dennoch vergleichbare Sätze 
konstruieren.

\subsection*{(Vereinfachtes) Beispiel für einen gödelisierten Satz}

Wieder ein vereinfachtes Beispiel: wir nehmen an, dass jeder mit den Grundzeichen
der Mathematik definierbare Satz eine Nummer bekommt. Dort sind nun alle nur überhaupt
definierbaren Sätze, sinnvoll wie sinnlos, falsch wie richtig,
wie z.\,B. \frq $45 + 16 = 1$\flq, \frq $1 + 1 = 2$\flq, aber auch \frq der Satz mit
der Nummer 4922\footnote{Die Nummern sind nur beispielhaft gewählt und entsprechen
nicht der wirklichen Gödelisierung der Sätze.}
ist nicht beweisbar\flq. Wenn aber letzterer Satz gerade 
\frq zufällig\flq\ tatsächlich die zugewiesene Nummer 4922 hat, dann behauptet
der Satz seine eigene Unbeweisbarkeit, und das mehr oder weniger
zufällig, ohne dass eine selbstreferenzielle-Strukturen-verbietende
Typentheorie das verbieten könnte.\mnote{Siehe dazu Fußnote \ref{zufaellig}.}

Unbeweisbar heißt hier, dass der Satz nicht hergeleitet werden kann aus den gewählten
Axiomen. Aber durch metamathematische Überlegungen können wir ihn doch als wahr anerkennen,
weil er eben nicht aus den Axiomen herleitbar ist (d.\,h. unbeweisbar, was der
Satz ja auch behauptet).

Das letzte Ressort könnte sein, ihn als Axiom zu setzen. Aber da alle potenziell
möglichen Sätze generiert werden, dürfen wir den Satz mit der Nummer 559321 nicht
aus den Augen verlieren, er besagt: \frq der Satz mit der Nummer 559321 ist
nicht beweisbar\flq\mnote{Dieser Satz könnte auch aussagen: $\thicksim \thicksim \thicksim \text{Bew}(4922) $, wenn
man mehr als die absolute Minimalmenge nimmt. Vgl. dazu Kommentar \ref{minimalmenge}. Aber das würde die
Sache nur komplizierter machen.}. Wir 
stehen also wieder vor dem gleichen Problem und müssten
wieder ein Axiom setzen. Das geht ewig so weiter. Wir haben (nach unendlich viel
vergangener Zeit und unendlich vielen neuen Axiomen) also wieder ein System,
das aber nicht mehr algorithmisch lösbar ist, weil Algorithmen, die etwas
lösen sollen, keine unendliche Menge an Axiomen abarbeiten können. 

Gödels Algorithmus bietet die Möglichkeit, jedem sinnvollen und sinnlosen 
Satz\mnote{Siehe dazu S. \pageref{grundzeichenzahlen}. Die gewählten Grundzeichen hängen
vom System ab, das man als Grundlage nehmen möchte. Gödel nimmt das System der Peano-Axiome
mit dem System der Principia Mathematica zusammen zur Vereinfachung seines Beweises.}
eine eineindeutige Nummer zuzuordnen, indem er einige Konstanten 
definiert, z.\,B. wird \frq 0\flq\ zu 1, \frq f\flq\ zu 3, 
\frq$\thicksim$\flq\ zu 5 usw. 
Außerdem nutzt er die Eineindeutigkeit von Primfaktorzerlegungen aus.\mnote{Jede
natürliche Zahl ist entweder selbst Prim oder das Produkt aus $n$ Primzahlen. Die Zerlegung
einer Zahl in Primzahlen ist eineindeutig und jede Zahl hat exakt eine ihr zugeordnete
Primfaktorzerlegung. Nimmt man die Position $p$ als $p$te Primzahl hoch einer Zahl, die
ein spezielles Zeichen symbolisiert, so lässt sich diese Zahl jederzeit zurückwandeln
in den ursprünglichen Satz, weil die Primfaktorzerlegung immer eineindeutig bleibt.}
So kann man den (sinnlosen) Satz: 
\frq es gilt nicht, dass 4\flq\ folgendermaßen umwandeln:

\begin{enumerate}
	\item \frq es gilt nicht, dass 4\flq
	\item \frq$\thicksim 4$\flq
	\item \frq$\thicksim ffff0$\flq\ (mit der peano'schen Nachfolgerfunktion $f$)\mnote{Siehe dazu § \refsatz{0 N x}}
	\item Nun nehmen wir die Position $p$ im String (angefangen von links mit 1)\mnote{Siehe dazu § \refsatz{Z(n)}}
		und nehmen die $p$te Primzahl als Basis ($p(p)$), die Ziffer für das Zeichen
		als Exponent: $p(1)^5 \times p(2)^3 \times p(3)^3 \times p(3)^3 \times p(4)^3 \times p(5)^1$.
		Wir erhalten:
		$2^5 \times 3^3 \times 5^3 \times 7^3 \times 11^3 \times 13^1 = 640972332000$. Dies
		können wir, da es unendlich viele Primzahlen gibt\mnote{\customlabel{eineindeutigkeit}{\arabic{commentaryNumber}}Die Eineindeutigkeit der Zuordnung von Zahlenwerten
		zu Zeichenketten bedeutet de facto, dass jede mögliche Zeichenkette existiert;
		diese Eineindeutigkeit Zahl $\longleftrightarrow$ Satz ist wichtig, denn in diesen unendlich
		vielen Sätzen wären alle Texte nach \frq mathematischer Grammatik\flq\ entstanden. 
		Vgl. dazu die Idee der \frq Library of Babel\flq, wo zwar (wahrscheinlich) nicht alle
		Texte gespeichert werden, sie aber im Bereich der mathematischen Hashfunktion,die
		angewandt wird, um die Seiten \frq statisch\flq\ zu erzeugen, dennoch bereits alle
		\frq existieren\flq; hierzu auch eine metaphysische Anmerkung: damit wird
		ein platonischer Realismus der mathematischen Welt vorausgesetzt, in der das Ergebnis jeder
		Gleichung schon \frq existiert\flq, unabhängig davon, ob ausgerechnet oder nicht. In
		einem z.\,B. ultrafinitistischen Denkmodell wäre Gödel gar kein Problem (womit ich nicht
		dafür argumentieren möchte, sondern eher Platonist bin).}, für 
		jede beliebig lange Zeichenkette 
		bestehend aus den Grundzeichen machen. Die Zahl 640972332000 lässt sich durch 
		Primfaktorzerlegung prinzipiell wieder 1:1 in den Ursprungssatz zurückwandeln, so dass gilt:

		$$ 640972332000 \equiv \text{\frq es gilt nicht, dass 4\flq} $$

		Die Zahl $640972332000$ ist damit eineindeutig ebendiesem Satz zugeordnet und mit dem
		gewählten Alphabet keinem anderen ($\longrightarrow$ Eineindeutigkeit).
\end{enumerate}

Der Begriff der Beweisbarkeit\mnote{Vgl. § \refsatz{Bw}}
lässt sich im System der Principia Mathematica selbst
darstellen als die Möglichkeit der Herleitung aus den Axiomen. Nehmen wir die Abkürzungen
von Gödel selbst, dann können wir die Beweisbarkeit eines Satzes folgendermaßen
darstellen: $\text{Bew}\left(x\right)$.\mnote{Vgl. § \refsatz{Bew}}
Wenn wir also einen Satz der Form:

$$
n \vdash \thicksim \text{Bew}\left(m\right)
$$
(wobei $n$ seine gödelisierte Satznummer und \frq zufälligerweise\flq\ $n = m$, 
d.\,h. eine lange Zahl, die eineindeutig dem
Satz zugeordnet ist, ist\footnote{Wie Gödel es selbst sagt, wäre es kein 
Problem,\mnote{\customlabel{luegeeins}{\arabic{commentaryNumber}} 
Kommentar \ref{luegezwei} ist richtig.} diese Zahl für jede Aussage tatsächlich 
aufzuschreiben; es wäre nur sehr umständlich, 
siehe Fußnote \ref{leichtaufzuschreiben}.}, der rein zufällig auf sich selbst verweist), 
dann haben wir einen solchen Satz, der seine eigene Unbeweisbarkeit
behauptet und im System nicht entscheidbar ist. Wäre er beweisbar, wäre er per definitionem
richtig, weil nur richtige Sätze beweisbar sind, aber wäre er beweisbar, wäre er auch 
per definitionem unbeweisbar. Seine Richtigkeit lässt sich aber eben durch
metamathematische Überlegungen zeigen.\mnote{Vgl. Kommentar \ref{metamatematischeueberlegungen}.}

\subsection*{\frq\textit{\dots und verwandter Syteme}\flq}

Für solche Sätze sind Systeme wie das der Principia Mathematica prinzipiell 
überflüssig,\mnote{Siehe wieder Fußnote \ref{leichtaufzuschreiben}}
weil diese nur auf den logischen Axiomen aufbauen und Abkürzungen für lange Reihen von
Zeichenketten darstellen. Daher sind alle logisch-formalen Systeme betroffen und die
Principia ist nur ein \frq Einstiegspunkt\flq\ zur Vereinfachung der Thesen.

Benötigt zum Erzeugen solcher Sätze sind nur die logischen Schlussregeln und die
potenzielle Möglichkeit, etwas wie natürliche Zahlen herzuleiten, sowie die Möglichkeit,
im System selbst herauszufinden, ob ein Satz ein Beweis eines anderen Satzes ist, ohne dabei
auf ein Metasystem zurückzugreifen.

\subsection*{$\omega$-Widerspruchsfreiheit}

Ein System, in dem die Axiome keine Widersprüchlichkeiten aufweisen, heißt
\frq widerspruchsfrei\flq. Aber nicht jedes widerspruchsfreie System ist auch
$\omega$-widerspruchsfrei. Die $\omega$-Widerspruchsfreiheit ist rigoroser als
die allgemeine Widerspruchsfreiheit, denn in der $\omega$-Widerspruchsfreiheit
gilt -- neben der allgemeinen Widerspruchsfreiheit --, dass auch keine kontraintuitiven
Schlüsse gezogen werden können dürfen (z.\,B. Sätze, die wahr sind, ohne dass 
ihre Wahrheit aus den Axiomen herleitbar ist).

Die Axiome der PM sind widerspruchsfrei, aber mit der Methode der Gödelisierung
lassen sich ebensolche Existenzaussagen treffen, die das System $\omega$-widersprüchlich
machen.\mnote{Dazu muss es nur \textit{irgendwie} möglich sein, 
aus dem System etwas wie die natürlichen Zahlen und Addition herleiten zu können, was auf fast alle
praktisch-relevanten logisch-widerspruchsfreien Systeme zutrifft. Dazu muss es auch möglich sein,
dass im genannten System der Begriff \frq die Formel $f$ ist beweisbar im 
System $S$\flq\ definierbar ist.} Dies gilt für alle ausreichend komplexen
Systeme. Wenn ein System ausreichend komplex ist, dann ist es notwendigerweise unvollständig,
d.\,h. es gibt Existenzaussagen, die nicht aus den Axiomen herleitbar sind (erster
Unvollständigkeitssatz)\mnote{Vgl. Satz VIII auf S. \pageref{satzvii}.}. Kann man jedoch die
Vollständigkeit des ausreichend-komplexen Systems beweisen, ist es notwendigerweise 
widersprüchlich (zweiter Unvollständigkeitssatz).\mnote{Vgl. Satz XI auf S. \pageref{satzxi}.}

\subsection*{Vollständigkeit und Entscheidbarkeit}

Ein formales System $P$ ist vollständig, wenn jede Aussage $p \in P$ entweder 
exakt \textit{wahr} oder \textit{falsch} ist. Wenn im System für jeden Wert
von $p$ der  Wahrheitswert algorithmisch anhand der Regeln des
formalen Systems bestimmt werden kann, heißt das System \frq entscheidbar\flq.

Es gibt Systeme, auf die die gödelschen Unvollständigkeitssätze nicht zutreffen,\mnote{Ohne
Primzahlen lassen sich die gödelschen Unvollständigkeitssatzbeweise 
nicht konstruieren, weil es keine sonstige Möglichkeit gibt, jedem Satz eine eineindeutige,
algorithmisch bestimmbare Nummer zuzuweisen.} beispielsweise die Presburger-Arithmetik; 
diese sind aber sehr eingeschränkt
(z.\,B. unterstützt die Presburger-Arithmetik keine Aussagen über Multiplikation,
Division oder Primzahlen).

\subsection*{Die beiden Unvollständigkeitssätze}

Aus der Arbeit Gödels folgen zwei fundamentale Sätze zur Metamathematik:

\begin{enumerate}
	\item Jedes ausreichend komplexe System beinhaltet immer Sätze, die innerhalb
		des Systems nicht bewiesen werden können.
	\item Jedes ausreichend komplexe System kann seine eigene Widerspruchsfreiheit
		nicht beweisen; wenn es das kann, dann ist es\mnote{Zweiter Unvollständigkeitssatz, 
		vgl. Kommentar \ref{zweiterunvollstaendigkeitssatz}.}
		notwendigerweise widersprüchlich.
\end{enumerate}

\subsection*{Auswirkungen}

Die Auswirkungen auf das Verständnis der Mathematik waren gewaltig, denn plötzlich war klar,
dass man nicht jede wahre Aussage auch beweisen kann. Aber auch für die Berechenbarkeitstheorie
und die später aufkommende Informatik sind betroffen. Gödels Ansatz, mathematische
Aussagen als Ansammlungen von nach Regeln gebildeter Zeichen zu sehen und nicht primär
inhaltlich zu deuten, hat die moderne Informatik überhaupt erst möglich gemacht.
Kein Computer kann mehr als ebendies: Zeichenketten nach genau vorgelegten Regeln
zu manipulieren, ohne dabei auch nur im Ansatz inhaltlich auf diese einzugehen. Gödel hat
gezeigt, dass es immer Aussagen gibt, die ein Computer nicht zurückführen kann auf
seine Axiome, d.\,h. dass es immer Aussagen gibt, die mit einem Computer nicht überprüfbar sind.
Das setzt -- ähnlich wie das Halteproblem -- der Theorie der Algorithmen sowie
der Beweistheorie überhaupt enge Grenzen.

\subsection*{Ziel dieser Arbeit}

Im Rahmen dieser Arbeit würde ich gern versuchen, den Originalalgorithmus soweit
es mir möglich ist nachzuvollziehen. Dazu habe ich die Arbeit von Gödel soweit ich 
es konnte exakt abgetippt und mit \LaTeX, einem Textsatzsystem für mathematische
Texte, komplett neu gesetzt, um einige Dinge wie eingefärbte passende Klammern zur besseren 
Lesbarkeit einzuführen und Kommentare hinzuzufügen, die die gödel'schen Formeln in eine modernere
Schreibweise übertragen. Trotz größtmöglicher Mühe und Bemühung um Genauigkeit
kann ich dabei natürlich nicht vermeiden, dass sich eventuell Tipp- oder Verständnisfehler 
eingeschlichen haben.

\subsection*{Lizenz}

Da es meiner Recherchen nach noch keine solche Version gab, habe ich sie der
Forschung auf meinem GitHub-Account unter der GPL2-Lizenz zur Verfügung 
gestellt\footnote{Siehe \url{https://github.com/NormanTUD/GoedelLaTex/}.}.
Jeder kann diese Version forken und eigene Änderungen hinzufügen, um das
Dokument möglichst vollständig zu machen und meine Fehler zu korrigieren.
Es unterliegt keinen Einschränkungen für Forschung und Lehre, außer dass 
(wie in der GPL2 mit \textit{Copyleft} üblich) jede Änderung selbst wieder
öffentlich gemacht und unter die GPL2 gestellt werden muss.

Juristisch beziehe ich mich auf §~3 des Urheberrechtsgesetzes, das eine
Bearbeitung und Kommentierung eines Werkes mit eigener Schöpfungshöhe
erlaubt.

\newpage
\pagenumbering{arabic}
\setcounter{footnote}{0}
\setcounter{page}{173}
\let\thefootnote\originalthefootnote
